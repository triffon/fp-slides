\documentclass[alsotrans]{beamerswitch}
\usepackage{fprog}

\title{Типове, видове, функтори и монади}

\date{20 януари 2020 г.}

\lstset{language=Haskell,style=Haskell}

\begin{document}

\begin{frame}
  \titlepage
\end{frame}

\section{Въведение}

\begin{frame}[fragile]
  \frametitle{Три свята}
  Във вселената на Haskell живеят всякакви интересни обитатели.\\
  Можете ли да назовете някои от тях?\pause\\[2ex]
  Обитателите са разпределени в три свята:\pause
  \begin{enumerate}[<+->]
    \small
  \item Светът на \textbf{стойностите}
    \begin{itemize}
    \item числови (\lst{1}, \lst{2.34}) 
    \item булеви (\lst{False}, \lst{True})
    \item списъчни и низови (\lst{[1, 2, 3]}, \lst{"Haskell"}, \lst{[]})
    \item функционални (\lst{sqrt}, \lst{(+)}, \lst{map}, \lst{\x -> x + 1})
    \end{itemize}
  \item Светът на \textbf{типовете}
    \begin{itemize}
    \item скаларни (\lst{Int}, \lst{Integer}, \lst{Float}, \lst{Bool})
    \item съставни (\lst{[Int]}, \lst{String}, \lst{[[Float]]})
    \item потребителски (\lst{Figure})
    \item функционални (\lst{Int -> Int}, \lst{(Int -> Bool) -> [Int] -> [Int]})
    \end{itemize}
  \item Светът на \textbf{видовете}
    \begin{itemize}
    \item \alert{?}
    \end{itemize}
  \end{enumerate}
\end{frame}

\begin{frame}[fragile]
  \frametitle{Правилата на вселената}
  \begin{itemize}[<+->]
    \small
  \item Световете са разделени на етажи (редове)
  \item Базовите стойности живеят на ред 0 в някой базов тип
    \begin{itemize}
    \item \lst{False :: Bool}
    \end{itemize}
  \item Функциите над стойности от ред $n$ са от ред $n+1$
    \begin{itemize}
    \item \lst!\f -> f 0 > 3!
    \end{itemize}
  \item Функциите над стойности живеят във функционални типове
    \begin{itemize}
    \item \lst{(Int -> Int) -> Bool}
    \end{itemize}
  \item Всички типове живеят на ред 0 във вида \lst{*}
    \begin{itemize}
    \item \lst{Bool :: *}
    \end{itemize}
  \item Функциите над типове от ред $n$ са от ред $n+1$
    \begin{itemize}
    \item \lst{[]}, \lst{[1, 2, 3] :: [Int] = ([] Int)}
    \item \lst{Tree}, \lst{Tree (Leaf 1.23) Empty :: (Tree Float)}
    \end{itemize}
  \item Функциите над типове живеят във функционални видове
    \begin{itemize}
    \item \lst{[], Tree :: * -> *}
    \item \lst{Container :: * -> * -> *}
    \end{itemize}
  \end{itemize}
\end{frame}

\begin{frame}
  \frametitle{На сафари}
  \small
  Видяхте ли къде са...
  \begin{itemize}[<+->]
  \item потребителските типове (\lst{Figure})?
    \begin{itemize}
    \item типове от ред 0 във вида \lst{*}
    \end{itemize}
  \item потребителските конструктори (\lst{Circle})?
    \begin{itemize}
    \item стойности във функционален тип (\lst{Float -> Figure})
    \end{itemize}
  \item инстанциите (\lst{Eq Figure})?
    \begin{itemize}
    \item живеят в специалния вид \lst{Constraint}
    \end{itemize}
  \item класовете (\lst{Eq})?
    \begin{itemize}
    \item функции с един типов аргумент във вида \lst{* -> Constraint}
    \end{itemize}
  \item параметричните типове (\lst{Tree}, \lst{PairsList})?
    \begin{itemize}
    \item функции над типове в някой функционален вид (\lst{* -> *}, \lst{* -> * -> *})
    \end{itemize}
  \end{itemize}
\end{frame}

\begin{frame}[fragile]
  \frametitle{Разновидности алгебрични типове}
  \small
  \begin{itemize}[<+->]
  \item Изброени типове (още: enums)
    \begin{itemize}
    \item \lst{data Bool = False | True}
    \item \lst{data Compare = LT | EQ | GT}
    \end{itemize}
  \item Записи (още: структури, records)
    \begin{itemize}
    \item \lst{data Player = Player String Int}
    \end{itemize}
  \item Алтернативи (още: обединения, unions, суми)
    \begin{itemize}
    \item \lst{data Figure = Circle Float | Rectangle Float Float}
    \end{itemize}
  \item Параметризирани типове (още: типови функции)
    \begin{itemize}
    \item \lst{data Container a = Empty | Box a}
    \item \lst{data Maybe a = None | Just a}
    \item \lst{data Either a b = Left a | Right b}
    \end{itemize}
  \item Рекурсивни типове (още: неподвижни точки)
    \vspace{-1ex}
    \begin{itemize}
    \item
\begin{lstlisting}
data Tree a = Empty | Leaf a |
              Tree { root :: a, left :: Tree a,
                                right :: Tree a}
\end{lstlisting}
    \end{itemize}
  \end{itemize}
\end{frame}
\end{document}
