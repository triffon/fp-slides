\documentclass{beamer}
\usepackage{fp}

\title{Списъци}

\date{4 ноември 2015 г.}

\begin{document}

\begin{frame}
  \titlepage
\end{frame}

\section{Точкови двойки}

\begin{frame}
  \frametitle{Точкови двойки}

  \begin{columns}[t,onlytextwidth]
    \column{0.5\textwidth}
    \vspace{1em}
    
    \centering
    \tt{(A . B)}
    
    \column{0.5\textwidth}

    \centering
    \begin{tabular}{cc}
      \hline
      \pointcell\\
      \hline
      \bda&\bda\\
      \fbox A &\fbox B
    \end{tabular}
  \end{columns}

  \pause
  \vspace{1em}

  \begin{itemize}[<+->]
  \item \tta{(cons }<израз$_1$> <израз$_2$>\tta)
  \item Точкова двойка от оценките на <израз$_1$> и <израз$_2$>
  \item \tta{(car }<израз>\tta)
  \item \textbf{Първият} компонент на двойката, която е оценката на <израз>
  \item \tta{(cdr }<израз>\tta)
  \item \textbf{Вторият} компонент на двойката, която е оценката на <израз>
  \item \tta{(pair? }<израз>\tta)
  \item Проверява дали оценката на <израз> е точкова двойка
  \end{itemize}

\end{frame}

\begin{frame}[fragile]
  \frametitle{Примери}

  \begin{columns}[t,onlytextwidth]

    \column{0.6\textwidth}

    \begin{tabular}{c}
      \tt{(cons (cons 2 3) (cons 8 13))}\\
      \bda\\
      \tt{((2 . 3) . (8 . 13))}
    \end{tabular}
    \vspace{4em}

\onslide<3->{
  \begin{tabular}{c}
    \tt{(cons 3 (cons (cons 13 21) 8))}\\
    \bda\\
    \tt{(3 . ((13 . 21) . 8))}
  \end{tabular}
}
    \column{0.4\textwidth}

\onslide<2->{
    \begin{tabular}{cc@{}c@{}cc}
      \cline{1-2}\cline{4-5}
      \pointcell&$\nextarrow$&\pointcell\\
      \cline{1-2}\cline{4-5}
      \bda&&&\bda&\bda\\
      \cline{1-2}
      \pointcell&&\fbox8&\fbox{13}\\
      \cline{1-2}
      \bda&\bda\\
      \fbox2 &\fbox 3
    \end{tabular}
    \vspace{1em}
}

\onslide<4>{
      \begin{tabular}{ccc@{}c@{}c}
      \cline{1-2}
      \pointcell\\
      \cline{1-2}
      \bda&\bda\\
      \cline{2-3}
      \fbox3&\pointcell&$\nextarrow$&\fbox8\\
      \cline{2-3}
      &\bda\\
      \cline{2-3}
      &\pointcell\\
      \cline{2-3}
      &\bda&\bda\\
      &\fbox{13}&\fbox{21}
    \end{tabular}
}
  \end{columns}
\end{frame}

\begin{frame}
  \frametitle{S-изрази}

  \begin{definition}
    S-израз наричаме:
    \begin{itemize}
    \item атоми (булеви, числа, знаци, символи, низове)
    \item точкови двойки \tt{(S$_1$ . S$_2$)}, където \tt{S$_1$} и \tt{S$_2$} са S-изрази
    \end{itemize}
  \end{definition}
  \vspace{1em}

  \pause

  \alert{S-изразите са най-общия тип данни в Scheme.}

  \vspace{1em}
  С тяхна помощ могат да се дефинират произволно сложни структури от данни.
\end{frame}

\begin{frame}
  \frametitle{All you need is $\lambda$ --- точкови двойки}
\end{frame}

\section{Списъци}

\begin{frame}
  \frametitle{Списъци в Scheme}
\end{frame}

\begin{frame}
  \frametitle{Вградени функции за списъци}
\end{frame}

\begin{frame}
  \frametitle{Форми на равенство в Scheme}
\end{frame}

\begin{frame}
  \frametitle{memq, \ldots}
\end{frame}

\begin{frame}
  \frametitle{Съкратени форми на \tt{car} и \tt{cdr}}
\end{frame}


\subsection{Рекурсия над списъци}


\begin{frame}
  \frametitle{Обхождане на списъци}
\end{frame}

\begin{frame}
  \frametitle{Конструиране на списъци}
\end{frame}

\section{Вложени списъци}

\begin{frame}
  \frametitle{Работа с вложени списъци}
\end{frame}

\begin{frame}
  \frametitle{Примери}
\end{frame}

\section{Функции от по-висок ред за списъци}

\begin{frame}
  \frametitle{\tt{map}}
\end{frame}

\begin{frame}
  \frametitle{\tt{filter}}
\end{frame}

\begin{frame}
  \frametitle{\tt{accumulate}}
\end{frame}

\begin{frame}
  \frametitle{\tt{accumulate-i}}
\end{frame}

\begin{frame}
  \frametitle{\tt{accumulate1}}
\end{frame}


\section{Функции с произволен брой аргументи}

\begin{frame}
  \frametitle{Дефиниция}
\end{frame}

\begin{frame}
  \frametitle{\tt{apply}}
\end{frame}

\begin{frame}
  \frametitle{\tt{eval}}
\end{frame}

\end{document}
