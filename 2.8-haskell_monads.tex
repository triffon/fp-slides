\documentclass[alsotrans,beameroptions={aspectratio=169}]{beamerswitch}
\usepackage{fprog}

\title{Приложения на функтори и монади}

\date{15 януари 2025 г.}

\lstset{language=Haskell,style=Haskell}

\begin{document}

\begin{frame}
  \titlepage
\end{frame}

\section{Приложения на Монади}

\begin{frame}[fragile]
  \frametitle{Maybe }

  \lst{data Maybe a = Nothing | Just a}

  \pause
  \begin{itemize}[<+->]
  \item Фокусиране върху „добрия“ случай
  \item \lst{m1 <*> m2} --- трябва едновременно да имат стойност (\lst{and})
  \item \lst{m1 <|> m2} --- трябва поне едно да има стойност (\list{or})
  \item \lst{m1 >>= (\x -> m2)} --- стойността на \lst{m2} зависи от стойността на \lst{m1}
  \item \textbf{Пример:} търсене на път в дърво
  \end{itemize}
\end{frame}

\begin{frame}[fragile]
  \frametitle{Error }

  \lst{data Error a = OK a | Error String}

  \pause
  \begin{itemize}[<+->]
  \item Възможност за прекратяване на изчислението с изключение
  \item \lst{m1 <*> m2} --- трябва едновременно да имат стойност (\lst{and})
  \item \lst{m1 <|> m2} --- трябва поне едно да има стойност (\list{or})
  \item \lst{m1 >>= (\x -> m2)} --- стойността на \lst{m2} зависи от стойността на \lst{m1}
  \item \textbf{Пример:} търсене на път в дърво
  \end{itemize}
\end{frame}

\end{document}
