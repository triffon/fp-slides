\documentclass[alsotrans]{beamerswitch}
\usepackage{fprog}

\title[Среди и процеси]{Модел на средите и изчислителни процеси}

\date{9 октомври 2019 г.}

\lstset{language=Scheme}

\begin{document}

\begin{frame}
  \titlepage
\end{frame}

\section{Модел на средите}

\begin{frame}
  \frametitle{Среди в Scheme}

  \begin{itemize}[<+->]
  \item Връзката между символите и техните оценки се записват в речник, който се нарича \textbf{среда}.
  \item Всеки символ има най-много една оценка в дадена среда.
  \item В даден момент могат да съществуват много среди.
  \item Символите винаги се оценяват в една конкретна среда.
  \item \alert{Символите могат да има различни оценки в различни среди.}
  \item При стартиране Scheme по подразбиране работи в \textbf{глобалната среда}.
  \item В глобалната среда са дефинирани символи за стандартни операции и функции.
  \end{itemize}
\end{frame}

\begin{frame}
  \frametitle{Пример за среда}

  \begin{columns}[T,onlytextwidth]
    \begin{column}{.5\textwidth}
      \begin{itemize}[<+->]
      \item \lst{(define a 8)}
      \item \evalstoerr r
      \item \lst{(define r 5)}
      \item \evalsto{(+ r 3)}8
      \item \lst{(define (f x) (* x r))}
      \item \evalsto{(f 3)}{15}
      \item \evalsto{(f r)}{25}
      \end{itemize}
    \end{column}

    % TODO: да се реализира с TikZ
    \begin{column}{.5\textwidth}
      \begin{envir}{E}
        \\\firstinenv \tt a&:&8
        \only<3->{
          \\\tt r&:&\tt 5
        }
        \only<5->{
          \\\tt f&:&\funcenv x{(* x r)}E
        }
      \end{envir}
    \end{column}
  \end{columns}
\end{frame}

\begin{frame}
  \frametitle{Функции и среди}

  \begin{itemize}[<+->]
  \item Всяка функция \tt f пази указател към средата \env E, в която е дефинирана.
  \item При извикване на \tt f:
    \begin{itemize}
    \item създава се нова среда \env{E_1}, която разширява \env E
    \item в \env{E_1} всеки символ означаващ формален параметър се свързва с оценката на фактическия параметър
    \item тялото на $f$ се оценява в \env{E_1}
    \end{itemize}
  \end{itemize}
\end{frame}

\begin{frame}
  \frametitle{Дърво от среди}
  \begin{itemize}[<+->]
  \item Всяка среда пази указател към своя ``родителска среда'', която разширява
  \item така се получава дърво от среди
  \item при оценка на символ в дадена среда \env E
    \begin{itemize}
    \item първо се търси оценката му в \env E
    \item ако символът не е дефиниран в \env E, се преминава към родителската среда
    \item при достигане на най-горната среда, ако символът не е дефиниран и в нея се извежда съобщение за грешка
    \end{itemize}
  \end{itemize}
\end{frame}

\begin{frame}
  \frametitle{Извикване на дефинирана функция}

  \begin{overlayarea}{\textwidth}{\textheight}
    \begin{columns}[T,onlytextwidth]
      \begin{column}{.5\textwidth}
        \begin{itemize}[<+->]
        \item \lst{(define r 5)}
        \item \lst{(define a 3)}
        \item \lst{(define (f x) (* x r))}
        \item
          \begin{tabular}[t]{lc}
            \inenv E&\lst{(f a)}\\
            \nxt{&\bda\\
            \inenv E &\lst{(f 3)}\\
            \nxt{&\bda\\
            \inenv {E_1} &\lst{(* x r)}\\
            \nxt{&\bda\\
                    &\tt{15}}}}
          \end{tabular}
        \end{itemize}
      \end{column}

      % TODO: да се реализира с TikZ
      \begin{column}{.5\textwidth}
        \begin{tabular}{c}
          \begin{envir}{E}
            \\\firstinenv\tt r&:&\tt 5 \only<2->{\\\tt a&:&\tt 3}
            \only<3->{\\\tt f&:&\funcenv x{(* x r)}E}
          \end{envir}
          \\
          \only<6->{
          \Bigg\uparrow\\
          \begin{envir}{E_1}
            \\\firstinenv\tt x&:&3
          \end{envir}}
        \end{tabular}
      \end{column}
    \end{columns}
  \end{overlayarea}
\end{frame}

\section{Рекурсия}

\subsection{Рекурсивни функции}

\begin{frame}
  \frametitle{Какво е рекурсия?}

  \pause
  \begin{center}
    \includegraphics[width=0.7\textwidth]{images/matroska.jpg}\\
    \imageAttr{Matryoshka dolls}{User:Fanghong (оригинал) и User:Gnomz007 (производна)}{https://commons.wikimedia.org/wiki/File:Russian-Matroshka_no_bg.jpg}{CC BY-SA 3.0}
  \end{center}
\end{frame}

\begin{frame}
  \frametitle{Какво е рекурсия?}

  \begin{center}
    \includegraphics[width=0.9\textwidth]{images/sierpinski.png}
  \end{center}
\end{frame}

\begin{frame}
  \frametitle{Какво е рекурсия?}

  \pause
  Повторение чрез позоваване на себе си\\[1em]
  \pause
  Рекурсивна функция: дефинира се чрез себе си\\
  \begin{equation*}
    n! = \left\{
    \begin{array}{l@{}l@{\qquad}l}
      1,&\text{ при }n = 0,&\textbf{(база)}\\
      n\cdot (n-1)!,&\text{ при }n > 0.&\textbf{(стъпка)}
    \end{array}\right.
  \end{equation*}\\[1em]
  \pause
  \begin{itemize}
  \item Дава се отговор на най-простата задача (база, дъно)
  \item Показва се как сложна задача се свежда към една или няколко по-прости задачи от същия вид (стъпка)
  \end{itemize}
\end{frame}


\begin{frame}
  \frametitle{Рекурсивни уравнения}

  Какво означава ``да дефинираме функция чрез себе си''?\\[1em]
  \pause
  Да разгледаме \emph{рекурсивното уравнение}, в което $F$ е неизвестно:
  \begin{equation*}
    F(n) =
    \underbrace{\begin{cases}
      1,&\text{ при }n = 0,\\
      n \cdot F(n-1),&\text{ при }n > 0.
    \end{cases}}_{\Gamma(F)(n)}
  \end{equation*}
  \pause
  \alert{$n!$ е ``най-малкото'' решение на уравнението $F = \Gamma(F)$.}
\end{frame}

\begin{frame}[fragile]
  \frametitle{Най-малка неподвижна точка}

  \begin{theorem}[Knaster-Tarski]
    Ако $\Gamma$ е изчислим оператор, то уравнението $F = \Gamma(F)$ има единствено най-малко решение $f$ \pause (\textbf{най-малка неподвижна точка на $\Gamma$}). \pause Нещо повече, решението точно съответства на рекурсивна програма пресмятаща $f$ чрез $\Gamma$.
  \end{theorem}
  \pause
\begin{lstlisting}
(define (fact n)
  (if (= n 0) 1
      (* n (fact (- n 1)))))
\end{lstlisting}
  \pause
  Кое е \textbf{най-малкото решение} на уравнението $F(x) = 1 + F(x-1)$?
  \pause
\lstinline{(define (f x) (+ 1 (f (- x 1)))}\\
\evalsto{(f 0)}?\\
  \pause
  $f$ е ``празната функция'', т.е. $\mathrm{dom}(f) = \emptyset$.
\end{frame}

\begin{frame}
  \frametitle{Операционна и денотационна семантика}

  Два подхода за описание на смисъла на функциите в Scheme.\\
  \pause
  Нека \tt{(define (f x) $\Gamma$[f])} е рекурсивно дефинирана функция.\\
  \pause
  \alert{Коя е математическата функция $f$, която се пресмята от \tt f?}\\[1em]
  \pause
  \textbf{Денотационна семантика}\\
  $f$ е най-малката неподвижна точка на уравнението $F = \Gamma(F)$.\\[1em]
  \pause
  \textbf{Операционна семантика}\\
  Разглеждаме редицата от последователни оценки на комбинации\\
  \tt{(f a)} $\rightarrow$ \tt{$\Gamma$[f][x $\mapsto$ a]} $\rightarrow\ldots$\\
  Ако стигнем до елемент \tt b, който не е комбинация, то $f(a) := b$.\\[1em]
  \pause
  \alert{Функциите в Scheme имат дуален, но еквивалентен смисъл:}
  \begin{itemize}
  \item решения на рекурсивни уравнения
  \item изчислителни процеси, генериращи се при оценка
  \end{itemize}
\end{frame}

\subsection{Рекурсивни процеси}

\begin{frame}
  \frametitle{Оценка на рекурсивна функция}

  \begin{center}
    \footnotesize
    \begin{tabular}{c}
      \lst{(fact 4)}\\
      \pause
      \nxt{\bda\\
      \alt<+->{\lst{(* 4 (fact 3))}}{\lst{(if (= 4 0) 1 (* 4 (fact (- 4 1))))}}\\
      \nxt{\bda\\
      \alt<+->{\lst{(* 4 (* 3 (fact 2)))}}{\lst{(* 4 (if (= 3 0) 1 (* 3 (fact (- 3 1)))))}}\\
      \nxt{\bda\\
      \alt<+->{\lst{(* 4 (* 3 (* 2 (fact 1))))}}{\lst{(* 4 (* 3 (if (= 2 0) 1 (* 2 (fact (- 2 1))))))}}\\
      \nxt{\bda\\
      \alt<+->{\lst{(* 4 (* 3 (* 2 (* 1 (fact 0)))))}}{\lst{(* 4 (* 3 (* 2 (if (= 1 0) 1 (* 1 (fact (- 1 1)))))))}}\\
      % TODO: да се добави разширението на дъното (if (= 0 0) 1 ...)
      \nxt{\bda\\
      \lst{(* 4 (* 3 (* 2 (* 1 1))))}\\
      \nxt{\bda\\
      \lst{(* 4 (* 3 (* 2 1)))}\\
      \nxt{\bda\\
      \lst{(* 4 (* 3 2))}\\
      \nxt{\bda\\
      \lst{(* 4 6)}\\
      \nxt{\bda\\
      24}}}}}}}}}
    \end{tabular}
  \end{center}
\end{frame}

% временно смаляване на шрифта за листинги
\lstset{basicstyle=\tiny\ttfamily}

\begin{frame}
  \frametitle{Оценка на рекурсивна функция в среда}

  {
  \tiny
  \begin{columns}[t,onlytextwidth]
    \begin{column}{.63\textwidth}
      \begin{tabular}{lc}
        \nxt{
        \inenv E&\lst{(fact 4)}\\
        \nxt{&\bda\\
        \inenv{E_1}&\alt<+->{\lst{(* 4 (fact 3))}}{\lst{(if (= n 0) 1 (* n (fact (- n 1))))}}\\
        \nxt{&\bda\\
        \inenv{E_2}&\alt<+->{\lst{(* 4 (* 3 (fact 2)))}}{\lst{(* 4 (if (= n 0) 1 (* n (fact (- n 1)))))}}\\
        \nxt{&\bda\\
        \inenv{E_3}&\alt<+->{\lst{(* 4 (* 3 (* 2 (fact 1))))}}{\lst{(* 4 (* 3 (if (= n 0) 1 (* n (fact (- n 1))))))}}\\
        \nxt{&\bda\\
        \inenv{E_4}&\alt<+->{\lst{(* 4 (* 3 (* 2 (* 1 (fact 0)))))}}{\lst{(* 4 (* 3 (* 2 (if (= n 0) 1 (* n (fact (- n 1)))))))}}\\
        \nxt{&\bda\\
        \alt<+->{\inenv{E_4}&\lst{(* 4 (* 3 (* 2 (* 1 1))))}}{\inenv{E_5}&\lst{(* 4 (* 3 (* 2 (* 1 (if (= n 0) 1 (* n (fact (- n 1))))))))}}\\
        \nxt{&\bda\\
        \inenv{E_3}&\lst{(* 4 (* 3 (* 2 1)))}\\
        \nxt{&\bda\\
        \inenv{E_2}&\lst{(* 4 (* 3 2))}\\
        \nxt{&\bda\\
        \inenv{E_1}&\lst{(* 4 6)}\\
        \nxt{&\bda\\
        \inenv E&24}}}}}}}}}}\\
        % фантомен ред за предодвратяване на подскачането
        % TODO: да се реализира с {overlayarea}
        &\phantom{\lst{(* 4 (* 3 (* 2 (* 1 (if (= n 0) 1 (* n (fact (- n 1))))))))}}
      \end{tabular}
    \end{column}
    % TODO: да се реализира с TikZ
    \begin{column}{.37\textwidth}
      \begin{tabular}{*{8}{c@{}}c}
        \multicolumn{9}c{
        \begin{envir}{E}
          \\\firstinenv &&\\[1pt]\hspace{6ex}\tt{fact}&:&\funcenv n\ldots E
        \end{envir}}
        \\
        \multicolumn 2c{\visible<2->{\Bigg\uparrow}}&
        \visible<8->{\Bigg\uparrow}&
        \multicolumn 3c{\visible<4->{\Bigg\uparrow}}&
        \visible<10->{\Bigg\uparrow}&
        \multicolumn 2c{\visible<6->{\Bigg\uparrow}}\\
        \multicolumn 2c{
        \visible<2->{
        \begin{envir}{E_1}
          \\\firstinenv\tt n&:&4
        \end{envir}}}&
        \visible<8->{\Bigg\vert}&
        \multicolumn 3c{
        \visible<4->{
        \begin{envir}{E_2}
          \\\firstinenv\tt n&:&3
        \end{envir}}}&
        \visible<10->{\Bigg\vert}&
        \multicolumn 2c{
        \visible<6->{
        \begin{envir}{E_3}
          \\\firstinenv\tt n&:&2
        \end{envir}}}\\
        \multicolumn 2c{}&
        \visible<8->{\Bigg\vert}&
        \multicolumn 3c{}&
        \visible<10->{\Bigg\vert}&
        \multicolumn 2c{}\\
        &
        \multicolumn 3c{
        \visible<8->{
        \hspace{5ex}
        \begin{envir}{E_4}
          \\\firstinenv\tt n&:&1
        \end{envir}}}&&
        \multicolumn 3c{
        \visible<10->{
        \hspace{2ex}
        \begin{envir}{E_5}
          \\\firstinenv\tt n&:&0
        \end{envir}}}&
      \end{tabular}
    \end{column}
  \end{columns}
  }\ \\[1em]
  \nxt{Линеен рекурсивен процес}
\end{frame}

% възстановяване на шрифта за листинги
\lstset{basicstyle=\ttfamily}

\subsection{Итеративни процеси}

\begin{frame}[fragile]
  \frametitle{Факториел с цикъл}

  \begin{overlayarea}{\textwidth}{.5\textheight}
  % TODO: да се разреди текста, за да се предодврати подскачането
  \begin{columns}[T,onlytextwidth]
    \begin{column}{.5\textwidth}
      Факториел на C++\\
%TODO: може ли да се използва lstlisting тук?
\begin{semiverbatim}
\only<3>\fbox{int fact(int n)} \{
  \only<4>\fbox{int r = 1;}
  for(\only<5>\fbox{int i = 1}; \only<6>\fbox{i <= n}; \only<7>\fbox{i++})
    \only<8>\fbox{r *= i;}
  \only<9>\fbox{return r;}
\}
\end{semiverbatim}
    \end{column}
    \pause
    \begin{column}{.5\textwidth}
      Превод на Scheme\\
%TODO: може ли да се използва lstlisting тук?
\begin{semiverbatim}
(define (for n \only<4,8>\fbox{r} \only<5,7>\fbox{i})
  (if \only<6>\fbox{(<= i n)}
      (for n \only<8>\fbox{(* r i)} \only<7>\fbox{(+ i 1)})
      \only<9>\fbox{r}))

\only<3>\fbox{(define (fact n)}
  (for n \only<4>\fbox{1} \only<5>\fbox{1}))
\end{semiverbatim}
    \end{column}
  \end{columns}
\end{overlayarea}
\end{frame}

\begin{frame}
  \frametitle{Оценка на итеративен факториел}

  \begin{center}
    \small
    \begin{tabular}{c}
      \nxt{\lst{(fact 4)}\\
      \nxt{\bda\\
      \lst{(for 4 1 1)}\\
      \nxt{\bda\\
      \alt<+->{\lst{(for 4 1 2)}}{\lst{(if (<= 1 4) (for 4 (* 1 1) (+ 1 1)) 1)}}\\
      \nxt{\bda\\
      \alt<+->{\lst{(for 4 2 3)}}{\lst{(if (<= 2 4) (for 4 (* 1 2) (+ 2 1)) 2)}}\\
      \nxt{\bda\\
      \alt<+->{\lst{(for 4 6 4)}}{\lst{(if (<= 3 4) (for 4 (* 2 3) (+ 3 1)) 6)}}\\
      \nxt{\bda\\
      \alt<+->{\lst{(for 4 24 5)}}{\lst{(if (<= 4 4) (for 4 (* 6 4) (+ 4 1)) 24)}}\\
      \nxt{\bda\\
      \alt<+->{\lst{24}}{\lst{(if (<= 5 4) (for 4 (* 24 5) (+ 5 1)) 24)}}}}}}}}}
    \end{tabular}
  \end{center}\ \\[1em]
  \nxt{Линеен итеративен процес}
\end{frame}

\begin{frame}<1-13>[label=iterenv]
  \frametitle{Оценка на итеративен факториел със среди}

  \begin{columns}[T,onlytextwidth]
    \begin{column}{.6\textwidth}
      \scriptsize
      \begin{tabular}{lc}
        \nxt{\inenv E&\lst{(fact 4)}\\
                     &\nxt{\bda\\
        \inenv{E_0}&\alt<+->{\tt{(for \alert<14>4 1 1)}}{\lst{(for n 1 1)}}\\
                     &\nxt{\bda\\
        \inenv{E_1}&\alt<+->{\tt{(for \alert<14>4 1 2)}}{\lst{(if (<= i n) (for n (* r i) (+ i 1)) r)}}\\
                     &\nxt{\bda\\
        \inenv{E_2}&\alt<+->{\tt{(for \alert<14>4 2 3)}}{\lst{(if (<= i n) (for n (* r i) (+ i 1)) r)}}\\
                     &\nxt{\bda\\
        \inenv{E_3}&\alt<+->{\tt{(for \alert<14>4 6 4)}}{\lst{(if (<= i n) (for n (* r i) (+ i 1)) r)}}\\
                     &\nxt{\bda\\
        \inenv{E_4}&\alt<+->{\tt{(for \alert<14>4 24 5)}}{\lst{(if (<= i n) (for n (* r i) (+ i 1)) r)}}\\
                     &\nxt{\bda\\
        \inenv{E_5}&\alt<+->{\lst{24}}{\lst{(if (<= i n) (for n (* r i) (+ i 1)) r)}}}}}}}}}\\
        % фантомен ред за предодвратяване на подскачането
                     &\phantom{\lst{(if (<= i n) (for n (* r i) (+ i 1)) r)}}
      \end{tabular}
    \end{column}

    % TODO: да се реализира с TikZ
    \begin{column}{.4\textwidth}
      \tiny
      \begin{tabular}{*{8}{c@{}}c}
        \multicolumn 9c{
        \visible<2->{
        \begin{envir}{E_0}
          \\\firstinenv \tt n&:&4
        \end{envir}}}
        \\
        \multicolumn 9c{\visible<2->{\big\downarrow}}\\
        \multicolumn 9c{
        \begin{envir}{E}
          \\\firstinenv &&\\[1pt]\tt{fact}&:&\funcenv n{(for n 1 1)}E\\
          \tt{for}&:&\funcenv{\alert<14>n r i}\ldots E
        \end{envir}}
        \\
        \multicolumn 2c{\visible<4->\bua}&
        \visible<10->\bua&
        \multicolumn 3c{\visible<6->\bua}&
        \visible<12->\bua&
        \multicolumn 2c{\visible<8->\bua}\\
        \multicolumn 2c{
        \visible<4->{
        \begin{envir}{E_1}
          \\\firstinenv\alert<14>{\tt n}&\alert<14>:&\alert<14>4\\
          \tt r&:&1\\
          \tt i&:&1
        \end{envir}}}&
        \visible<10->{
        \begin{tabular}{@{}c@{}}
        \!\!\Bigg\vert\\[0pt]
        \!\!\bigg\vert
        \end{tabular}}&
        \multicolumn 3c{
        \visible<6->{
        \begin{envir}{E_2}
          \\\firstinenv\alert<14>{\tt n}&\alert<14>:&\alert<14>4\\
          \tt r&:&1\\
          \tt i&:&2
        \end{envir}}}&
        \visible<12->{
        \begin{tabular}{@{}c@{}}
        \!\!\Bigg\vert\\[0pt]
        \!\!\bigg\vert
        \end{tabular}}&
        \multicolumn 2c{
        \visible<8->{
        \begin{envir}{E_3}
          \\\firstinenv\alert<14>{\tt n}&\alert<14>:&\alert<14>4\\
          \tt r&:&2\\
          \tt i&:&3
        \end{envir}}}\\
        \multicolumn 2c{}&
        \visible<10->{\big\vert}&
        \multicolumn 3c{}&
        \visible<12->{\big\vert}&
        \multicolumn 2c{}\\
        &
        \multicolumn 3c{
        \visible<10->{
        \hspace{5ex}
        \begin{envir}{E_4}
          \\\firstinenv\alert<14>{\tt n}&\alert<14>:&\alert<14>4\\
          \tt r&:&6\\
          \tt i&:&4
        \end{envir}}}&&
        \multicolumn 3c{
        \visible<12->{
        \hspace{2ex}
        \begin{envir}{E_5}
          \\\firstinenv\alert<14>{\tt n}&\alert<14>:&\alert<14>4\\
          \tt r&:&24\\
          \tt i&:&5
        \end{envir}}}&
      \end{tabular}
    \end{column}
  \end{columns}
\end{frame}

\begin{frame}<1-2>[fragile,label=reciter]
  \frametitle{Рекурсивен и итеративен процес}

  \begin{overlayarea}{\textwidth}{\textheight}
    \begin{columns}[T,onlytextwidth]
      \begin{column}{.5\textwidth}
        \begin{center}
          \tiny
          \begin{tabular}{c}
            \tt{(fact 4)} \\
            \bda\\
            \tt{(* 4 (fact 3))}\\
            \bda\\
            \tt{(* 4 (* 3 (fact 2)))}\\
            \bda\\
            \tt{(* 4 (* 3 (* 2 (fact 1))))}\\
            \bda\\
            \tt{(* 4 (* 3 (* 2 (* 1 (fact 0)))))}\\
            \bda\\
            \tt{(* 4 (* 3 (* 2 (* 1 1))))}\\
            \bda\\
            \tt{(* 4 (* 3 (* 2 1)))}\\
            \bda\\
            \tt{(* 4 (* 3 2))}\\
            \bda\\
            \tt{(* 4 6)}\\
            \bda\\
            24
          \end{tabular}
        \end{center}
        \scriptsize
\begin{semiverbatim}
    (define (fact n)
      (if (= n 0) 1
          \only<2>\fbox{(* n }(fact (- n 1)))))
\end{semiverbatim}
      \end{column}
      \begin{column}{.5\textwidth}
        \begin{center}
          \tiny
          \begin{tabular}{c}
            \tt{(fact 4)}\\
            \bda\\
            \tt{(for \alert<3> 4 1 1)}\\
            \bda\\
            \tt{(for \alert<3> 4 1 2)}\\
            \bda\\
            \tt{(for \alert<3> 4 2 3)}\\
            \bda\\
            \tt{(for \alert<3> 4 6 4)}\\
            \bda\\
            \tt{(for \alert<3> 4 24 5)}\\
            \bda\\
            \tt{24}
          \end{tabular}
        \end{center}
        \scriptsize
\begin{semiverbatim}
(define (for \alert<3>n r i)
  (if (<= i n)
      (for \alert<3>n (* r i) (+ i 1))
      r))

(define (fact n)
  (for \alert<3>n 1 1))
\end{semiverbatim}
      \end{column}
    \end{columns}
  \end{overlayarea}
\end{frame}

\begin{frame}
  \frametitle{Опашкова рекурсия}

  \begin{itemize}[<+->]
  \item Функциите, в които има отложени операции генерират същински \textbf{рекурсивни процеси}
  \item Рекурсивно извикване, при което няма отложена операция се нарича \textbf{опашкова рекурсия}
  \item Функциите, в които всички рекурсивни извиквания са опашкови генерират \textbf{итеративни процеси}
  \item При итеративните процеси липсва етап на свиването на рекурсията
  \item Опашковата рекурсия се използва за симулиране на цикли
  \item В Scheme опашковата рекурсия \alert{по стандарт} се интерпретира като цикъл
    \begin{itemize}
    \item т.е. не се заделя памет за всяко рекурсивно извикване
    \end{itemize}
  \end{itemize}
\end{frame}

\section{Вложени дефиниции}

\againframe<3>{reciter}

\againframe<14>{iterenv}

\subsection{Влагане на \tt{define}}

\begin{frame}[fragile]
  \frametitle{Вложени дефиниции}

  \begin{itemize}[<+->]
  \item \tta{(define (}<функция> \{<параметър\}\tta) \{<дефиниция>\} <тяло>\tta)
  \item При извикване на <функция> първо се оценяват всички <дефиниция> и след това се оценява <тяло>
  \item Вложените дефиниции се оценяват и записват в средата, която се \textbf{оценява} функцията, а не в средата, в която е \textbf{дефинирана}
  \item \textbf{Пример:}\\
\begin{lstlisting}
(define (dist x1 y1 x2 y2)
  (define dx (- x2 x1))
  (define dy (- y2 y1))
  (define (sq x) (* x x))
  (sqrt (+ (sq dx) (sq dy))))
\end{lstlisting}
  \end{itemize}
\end{frame}

\begin{frame}
  \frametitle{Оценка на вложени функции}

  \scriptsize
  \begin{overlayarea}{\textwidth}{\textheight}
    \begin{columns}[T,onlytextwidth]
      \begin{column}{.5\textwidth}
        \begin{tabular}{rc}
          \nxt{\inenv E&\lst{(dist 2 5 -1 9)}\\
                       &\nxt{\nxt{\bda\\
          \inenv{E_1}&\lst{(define dx (- x2 x1))}\\
          \nxt{\inenv{E_1}&\lst{(define dy (- y2 y1))}\\
          \nxt{\inenv{E_1}&\lst{(define (sq x) (* x x))}\\
          \nxt{\inenv{E_1}&\lst{(sqrt (+ (sq dx) (sq dy)))}\\
                       &\nxt{\bda\\
          \inenv{E_2}&\lst{(sqrt (+ (* x x) (sq dy)))}\\
                       &\nxt{\bda\\
          \inenv{E_3}&\lst{(sqrt (+ 9 (* x x)))}\\
                       &\nxt{\bda\\
          \inenv{E_1}&\lst{(sqrt (+ 9 16))}\\
                       &\nxt{\bda\\
          \inenv{E_1}&\lst{(sqrt 25)}\\
                       &\nxt{\bda\\
          \inenv{E_1}&\tt 5}}}}}}}}}}}
        \end{tabular}
      \end{column}
      % TODO: да се реализира с TikZ
      \begin{column}{.5\textwidth}
        \begin{tabular}{cc}
          \multicolumn 2c{
          \begin{envir}E
            \\\firstinenv \tt{dist}&:&\funcenv{x1 y1 x2 y2}\ldots E
          \end{envir}}
          \\
          \multicolumn 2c{\visible<2->\bua}
          \\
          \multicolumn 2c{\visible<2->{
          \begin{envir}{E_1}
            \\\firstinenv \tt{x1}&:&2 \\\tt{y1}&:&5 \\\tt{x2}&:&-1
            \\\tt{y2}&:&9 \only<3->{\\\tt{dx}&:&-3}
            \only<4->{\\\tt{dy}&:&4} \only<5->{\\\tt{sq}&:&\funcenv x{(*
                x x)}{E_1}}
          \end{envir}}}
          \\
          \only<7->\bua&\visible<8->\bua\\
          \only<7->{
          \begin{envir}{E_2}
            \\\firstinenv \tt x&:&3
          \end{envir}}& \only<8->{
                        \begin{envir}{E_3}
                          \\\firstinenv \tt x&:&4
                        \end{envir}}
        \end{tabular}
      \end{column}
    \end{columns}
  \end{overlayarea}
\end{frame}

\begin{frame}[fragile]
  \frametitle{Вложена помощна итеративна функция}

  При итеративни функция е удобно помощната функция да е вложена.
  \begin{overlayarea}{\textwidth}{.5\textheight}
    \begin{onlyenv}<1| handout:0>
\begin{lstlisting}
(define (for n r i)
  (if (<= i n)
      (for n (* r i) (+ i 1))
      r))

(define (fact n)
  (for n 1 1))
\end{lstlisting}
    \end{onlyenv}
    \begin{onlyenv}<2->
\begin{lstlisting}
(define (fact n)
  (define (for r i)
    (if (<= i n)
        (for (* r i) (+ i 1))
        r))
  (for 1 1))
\end{lstlisting}
    \end{onlyenv}
  \end{overlayarea}
  \onslide<3->
  Вложените дефиниции ``виждат'' символите на обхващащите им дефиниции.
\end{frame}

\begin{frame}
  \frametitle{Оценка на итеративен факториел с вложена функция}

  \begin{columns}[T,onlytextwidth]
    \begin{column}{.6\textwidth}
      \scriptsize
      \begin{tabular}{lc}
        \nxt{\inenv E&\lst{(fact 4)}\\
                     &\nxt{\bda\\
        \inenv{E_0}&\alt<+->{\lst{(for 1 1)}}{\lst{(define (for r i) ...)}}\\
                     &\nxt{\bda\\
        \inenv{E_1}&\alt<+->{\lst{(for 1 2)}}{\lst{(if (<= i n) (for (* r i) (+ i 1)) r)}}\\
                     &\nxt{\bda\\
        \inenv{E_2}&\alt<+->{\lst{(for 2 3)}}{\lst{(if (<= i n) (for (* r i) (+ i 1)) r)}}\\
                     &\nxt{\bda\\
        \inenv{E_3}&\alt<+->{\lst{(for 6 4)}}{\lst{(if (<= i n) (for (* r i) (+ i 1)) r)}}\\
                     &\nxt{\bda\\
        \inenv{E_4}&\alt<+->{\lst{(for 24 5)}}{\lst{(if (<= i n) (for (* r i) (+ i 1)) r)}}\\
                     &\nxt{\bda\\
        \inenv{E_5}&\alt<+->{\lst{24}}{\lst{(if (<= i n) (for (* r i) (+ i 1)) r)}}}}}}}}}\\
        % фантомен ред за предодвратяване на подскачането
        % подскачането да се предодврати с {overlayarea}
        &\phantom{\lst{(if (<= i n) (for (* r i) (+ i 1)) r)}}
      \end{tabular}
    \end{column}
    % TODO: да се реализира с TikZ
    \begin{column}{.4\textwidth}
      \tiny
      \begin{tabular}{*{8}{c@{}}c}
        \multicolumn 9c{
        \begin{envir}{E}
          \\\firstinenv &&\\[1pt]\tt{fact}&:&\funcenv n{(for 1 1)}E
        \end{envir}}
        \\
        \multicolumn 9c{\visible<2->\bua}
        \\
        \multicolumn 9c{
        \visible<2->{
        \begin{envir}{E_0}
          \\\firstinenv \tt n&:&4\\
          \hspace{6ex}\tt{for}&:&\funcenv{r i}\ldots {\alert<2>{E_0}}
        \end{envir}}}
        \\
        \multicolumn 2c{\visible<4->\bua}&
        \visible<10->\bua&
        \multicolumn 3c{\visible<6->\bua}&
        \visible<12->\bua&
        \multicolumn 2c{\visible<8->\bua}\\
        \multicolumn 2c{
        \visible<4->{
        \begin{envir}{E_1}
          \\\firstinenv\tt r&:&1\\
          \tt i&:&1
        \end{envir}}}&
        \visible<10->{
        \begin{tabular}{@{}c@{}}
        \!\!\Bigg\vert\\[0pt]
        \!\!\big\vert
        \end{tabular}}&
        \multicolumn 3c{
        \visible<6->{
        \begin{envir}{E_2}
          \\\firstinenv\tt r&:&1\\
          \tt i&:&2
        \end{envir}}}&
        \visible<12->{
        \begin{tabular}{@{}c@{}}
        \!\!\Bigg\vert\\[0pt]
        \!\!\big\vert
        \end{tabular}}&
        \multicolumn 2c{
        \visible<8->{
        \begin{envir}{E_3}
          \\\firstinenv\tt r&:&2\\
          \tt i&:&3
        \end{envir}}}\\
        \multicolumn 2c{}&
        \visible<10->{\big\vert}&
        \multicolumn 3c{}&
        \visible<12->{\big\vert}&
        \multicolumn 2c{}\\
        &
        \multicolumn 3c{
        \visible<10->{
        \hspace{5ex}
        \begin{envir}{E_4}
          \\\firstinenv\tt r&:&6\\
          \tt i&:&4
        \end{envir}}}&&
        \multicolumn 3c{
        \visible<12->{
        \hspace{2ex}
        \begin{envir}{E_5}
          \\\firstinenv\tt r&:&24\\
          \tt i&:&5
        \end{envir}}}&
      \end{tabular}
    \end{column}
  \end{columns}
\end{frame}

\subsection{\tt{let} и \tt{let*}}

\begin{frame}
  \frametitle{Специална форма \tt{let}}

  \begin{itemize}[<+->]
  \item \tta{(let} \tta(\{\tta({}<символ> <израз>\tta)\}\tta) <тяло>\tt)
  \item \tta{(let ((}{}<символ$_1$> <израз$_1$>\tta)\\
    \tta{\hskip 7ex(}{}<символ$_2$> <израз$_2$>\tta)\\
    \hskip 7ex\ldots\\
    \tta{\hskip 7ex(}{}<символ$_n$> <израз$_n$>\tta{))}\\
    \hskip 7ex{}<тяло>\tta)
  \item При оценка на \tt{let} в среда \env E:
    \begin{itemize}[<+->]
    \item Създава се нова среда \env{E_1} разширение на текущата
      среда \env E
    \item Оценката на <израз$_1$> в \env E се свързва със <символ$_1$> в \env{E_1}
    \item Оценката на <израз$_2$> в \env E се свързва със <символ$_2$> в \env{E_1}
    \item \ldots
    \item Оценката на <израз$_n$> в \env E се свързва със <символ$_n$> в \env{E_1}
    \item Връща се оценката на <тяло> в средата \env{E_1}
    \end{itemize}
  \item \alert{\tt{let} няма странични ефекти върху средата!}
    \begin{itemize}
    \item за разлика от \tt{define}
    \end{itemize}
  \end{itemize}
\end{frame}

\begin{frame}<1-3>[fragile]
  \frametitle{Пример за \tt{let}}

\begin{lstlisting}
(define (dist x1 y1 x2 y2)
  (let ((dx (- x2 x1))
        (dy (- y2 y1)))
   (sqrt (+ (sq dx) (sq dy)))))
\end{lstlisting}
\pause
\begin{lstlisting}
(define (area x1 y1 x2 y2 x3 y3)
  (let ((a (dist x1 y1 x2 y2))
        (b (dist x2 y2 x3 y3))
        (c (dist x3 y3 x1 y1))
        @\alert<3>{(p (/ (+ a b c) 2))})@
   (sqrt (* p (- p a) (- p b) (- p c)))))
\end{lstlisting}
\end{frame}

\begin{frame}
  \frametitle{Оценка на \tt{let}}

  % TODO: примерът да бъде за area, понеже там се вижда разлика между let и let*
  \scriptsize
  \begin{columns}[T,onlytextwidth]
    \begin{column}{.5\textwidth}
      \begin{tabular}{rc}
        \nxt{\inenv E&\lst{(dist 2 5 -1 9)}\\
                     &\nxt{\bda\\
        \inenv{E_1}&\begin{tabular}[t]{l}
                      \lst{(let ((dx (- x2 x1))}\\
                      \hskip 7ex\lst{(dy (- y2 y1)))}\\
                      \hskip 2ex\lst{(sqrt (+ (sq dx) (sq dy))))}
                    \end{tabular}\\
                     &\nxt{\bda\\
        \inenv{E_2}&\lst{(sqrt (+ (sq dx) (sq dy)))}\\
                     &\nxt{\bda\\
        \inenv{E_3}&\lst{(sqrt (+ (* x x) (sq dy)))}\\
                     &\nxt{\bda\\
        \inenv{E_4}&\lst{(sqrt (+ 9 (* x x)))}\\
                     &\nxt{\bda\\
        \inenv{E_2}&\lst{(sqrt (+ 9 16))}\\
                     &\nxt{\bda\\
        \inenv{E_2}&\lst{(sqrt 25)}\\
                     &\nxt{\bda\\
        \inenv{E_2}&\tt 5}}}}}}}}
      \end{tabular}
    \end{column}
    % TODO: да се реализира с TikZ
    \begin{column}{.5\textwidth}
      \begin{tabular}{ccc}
        \multicolumn 3c{
        \begin{envir}E
          \\\firstinenv \tt{dist}&:&\funcenv{x1 y1 x2 y2}\ldots E
        \end{envir}}
        \\
        \onslide<4->{\bua}&\only<2->\bua&\onslide<5->{\bua}\\
        \onslide<4->{
        \begin{envir}{E_3}
          \\\firstinenv \tt x&:&3
          % фантомни редове за да няма дупки между стрелките и таблицата
          \\&&\phantom{\tt y}
          \\&&\phantom{\tt y}
          \\&&\phantom{\tt y}
        \end{envir}}& \onslide<2->{
                      \begin{envir}{E_1}
                        \\\firstinenv \tt{x1}&:&2 \\\tt{y1}&:&5
                        \\\tt{x2}&:&-1 \\\tt{y2}&:&9
                      \end{envir}}& \onslide<5->{
                                    \begin{envir}{E_4}
                                      \\\firstinenv \tt x&:&4
                                      % фантомни редове за да няма дупки между стрелките и таблицата
                                      \\&&\phantom{\tt y}
                                      \\&&\phantom{\tt y}
                                      \\&&\phantom{\tt y}
                                    \end{envir}}\\
        \multicolumn 3c{\visible<3->\bua}\\
        \multicolumn 3c{\visible<3->{
        \begin{envir}{E_2}
          \\\firstinenv \tt{dx}&:&-3 \\\tt{dy}&:&4
        \end{envir}}}
      \end{tabular}
    \end{column}
  \end{columns}
\end{frame}

\begin{frame}
  \frametitle{Специална форма \tt{let*}}

  \begin{itemize}[<+->]
  \item \tta{(let*} \tta(\{\tta({}<символ> <израз>\tta)\}\tta) <тяло>\tt)
  \item \tta{(let* ((}{}<символ$_1$> <израз$_1$>\tta)\\
    \tta{\hskip 8ex(}{}<символ$_2$> <израз$_2$>\tta)\\
    \hskip 7ex\ldots\\
    \tta{\hskip 8ex(}{}<символ$_n$> <израз$_n$>\tta{))}\\
    \tta{\hskip 8ex}{}<тяло>\tta)
  \item При оценка на \tt{let*} в среда \env E:
    \begin{itemize}[<+->]
    \item Създава се нова среда \env{E_1} разширение на текущата
      среда \env E
    \item Оценката на <израз$_1$> в \env E се свързва със <символ$_1$> в \env{E_1}
    \item Създава се нова среда \env{E_2} разширение на текущата
      среда \env{E_1}
    \item Оценката на <израз$_2$> в \env{E_1} се свързва със <символ$_2$> в \env{E_2}
    \item \ldots
    \item Създава се нова среда \env{E_n} разширение на текущата
      среда \env{E_{n-1}}
    \item Оценката на <израз$_n$> в \env{E_{n-1}} се свързва със <символ$_n$> в \env{E_n}
    \item Връща се оценката на <тяло> в средата \env{E_n}
    \end{itemize}
  \end{itemize}
\end{frame}

\begin{frame}[fragile]
  \frametitle{Пример за \tt{let*}}

\begin{lstlisting}
(define (area x1 y1 x2 y2 x3 y3)
  (let* ((a (dist x1 y1 x2 y2))
         (b (dist x2 y2 x3 y3))
         (c (dist x3 y3 x1 y1))
         (p (/ (+ a b c) 2)))
\end{lstlisting}
  \pause
  \alert{Редът има значение!}
\begin{lstlisting}
(define (area x1 y1 x2 y2 x3 y3)
  (let* (@\alert{(p (/ (+ a b c) 2))}@
         (a (dist x1 y1 x2 y2))
         (b (dist x2 y2 x3 y3))
         (c (dist x3 y3 x1 y1)))
   (sqrt (* p (- p a) (- p b) (- p c)))))
\end{lstlisting}
\end{frame}

\begin{frame}
  \frametitle{Оценка на \tt{let*}}

  \scriptsize
  % TODO: примерът да бъде за area, понеже там се вижда разлика между let и let*
  \begin{columns}[T,onlytextwidth]
    \begin{column}{.5\textwidth}
      \begin{tabular}{rc}
        \nxt{\inenv E&\lst{(dist 2 5 -1 9)}\\
                     &\nxt{\bda\\
        \inenv{E_1}&\begin{tabular}[t]{l}
                      \lst{(let* ((dx (- x2 x1))}\\
                      \hskip 8ex\lst{(dy (- y2 y1)))}\\
                      \hskip 2ex\lst{(sqrt (+ (sq dx) (sq dy))))}
                    \end{tabular}\\
                     &\nxt{\nxt{\bda\\
        \inenv{E_3}&\lst{(sqrt (+ (sq dx) (sq dy)))}\\
                     &\nxt{\bda\\
        \inenv{E_4}&\lst{(sqrt (+ (* x x) (sq dy)))}\\
                     &\nxt{\bda\\
        \inenv{E_5}&\lst{(sqrt (+ 9 (* x x)))}\\
                     &\nxt{\bda\\
        \inenv{E_3}&\lst{(sqrt (+ 9 16))}\\
                     &\nxt{\bda\\
        \inenv{E_3}&\lst{(sqrt 25)}\\
                     &\nxt{\bda\\
        \inenv{E_3}&\tt5}}}}}}}}}
      \end{tabular}
    \end{column}
    % TODO: да се реализира с TikZ
    \begin{column}{.5\textwidth}
      \begin{tabular}{ccc}
        \multicolumn 3c{
        \begin{envir}E
          \\\firstinenv \tt{dist}&:&\funcenv{x1 y1 x2 y2}\ldots E
        \end{envir}}
        \\
        \onslide<5->{\bua}&\only<2->\bua&\onslide<6->{\bua}\\
        \onslide<5->{
        \begin{envir}{E_4}
          \\\firstinenv \tt x&:&3
          % фантомни редове за да няма дупки между стрелките и таблицата
          \\&&\phantom{\tt y}
          \\&&\phantom{\tt y}
          \\&&\phantom{\tt y}
        \end{envir}}&
                      \onslide<2->{
                      \begin{envir}{E_1}
                        \\\firstinenv \tt{x1}&:&2
                        \\\tt{y1}&:&5
                        \\\tt{x2}&:&-1
                        \\\tt{y2}&:&9
                      \end{envir}}&
                                    \onslide<6->{
                                    \begin{envir}{E_5}
                                      \\\firstinenv \tt x&:&4
                                      % фантомни редове за да няма дупки между стрелките и таблицата
                                      \\&&\phantom{\tt y}
                                      \\&&\phantom{\tt y}
                                      \\&&\phantom{\tt y}
                                    \end{envir}}\\
        \multicolumn 3c{\visible<3->\bua}\\
        \multicolumn 3c{\visible<3->{
        \begin{envir}{E_2}
          \\\firstinenv \tt{dx}&:&-3
        \end{envir}}}\\
        \multicolumn 3c{\visible<4->\bua}\\
        \multicolumn 3c{\visible<4->{
        \begin{envir}{E_3}
          \\\firstinenv \tt{dy}&:&4
        \end{envir}}}
      \end{tabular}
    \end{column}
  \end{columns}
\end{frame}

\section{Нелинейни изчислителни процеси}

\subsection{Логаритмични процеси}

\begin{frame}[fragile]
  \frametitle{Степенуване}

  Функцията $x^n$ може да се дефинира по следния начин:
  \begin{equation*}
    x^n = \begin{cases}
      1,&\text{ ако }n = 0,\\
      \frac 1 {x^{-n}},&\text{ ако }n < 0,\\
      x\cdot x^{n-1},&\text{ ако }n > 0.
    \end{cases}
  \end{equation*}
  \pause
\begin{lstlisting}
(define (pow x n)
  (cond ((= n 0) 1)
        ((< n 0) (/ 1 (pow x (- n))))
        (else (* x (pow x (- n 1))))))
\end{lstlisting}
\end{frame}

\begin{frame}
  \frametitle{Оценка на степенуване}

  \begin{center}
    \tiny
    \begin{tabular}{c}
      \tt{(pow 2 6)}\\
      \nxt{\bda\\
      \tt{(* 2 (pow 2 5))}\\
      \nxt{\bda\\
      \tt{(* 2 (* 2 (pow 2 4)))}\\
      \nxt{\bda\\
      \tt{(* 2 (* 2 (* 2 (pow 2 3))))}\\
      \nxt{\bda\\
      \tt{(* 2 (* 2 (* 2 (* 2 (pow 2 2)))))}\\
      \nxt{\bda\\
      \tt{(* 2 (* 2 (* 2 (* 2 (* 2 (pow 2 1))))))}\\
      \nxt{\bda\\
      \tt{(* 2 (* 2 (* 2 (* 2 (* 2 (* 2 (pow 2 0)))))))}\\
      \nxt{\bda\\
      \tt{(* 2 (* 2 (* 2 (* 2 (* 2 (* 2 1))))))}\\
      \nxt{\bda\\
      \tt{(* 2 (* 2 (* 2 (* 2 (* 2 2)))))}\\
      \nxt{\bda\\
      \tt{(* 2 (* 2 (* 2 (* 2 4))))}\\
      \nxt{\bda\\
      \tt{(* 2 (* 2 (* 2 8)))}\\
      \nxt{\bda\\
      \tt{(* 2 (* 2 16))}\\
      \nxt{\bda\\
      \tt{(* 2 32)}\\
      \nxt{\bda\\
      \tt{64}}}}}}}}}}}}}}
    \end{tabular}
  \end{center}\ \\
  \nxt{Линеен рекурсивен процес}
\end{frame}

\begin{frame}[fragile]
  \frametitle{Бързо степенуване}

  Алтернативна дефиниция на $x^n$:
  \begin{equation*}
    x^n = \begin{cases}
      1,&\text{ ако }n = 0,\\
      \frac 1 {x^{-n}},&\text{ ако }n < 0,\\
      (x^{\frac n2})^2,&\text{ ако }n > 0, n\text{ --- четно},\\
      x\cdot x^{n-1},&\text{ ако }n > 0, n\text{ --- нечетно}.
    \end{cases}
  \end{equation*}
  \pause
\begin{lstlisting}
(define (qpow x n)
  (define (sqr x) (* x x))
  (cond ((= n 0) 1)
        ((< n 0) (/ 1 (qpow x (- n))))
        ((even? n) (sqr (qpow x (quotient n 2))))
        (else (* x (qpow x (- n 1))))))
\end{lstlisting}
\end{frame}

\begin{frame}
  \frametitle{Оценка на бързо степенуване}

  \begin{center}
    \scriptsize
    \begin{tabular}{c}
      \tt{(qpow 2 6)}\\
      \nxt{\bda\\
      \tt{(sqr (qpow 2 3))}\\
      \nxt{\bda\\
      \tt{(sqr (* 2 (qpow 2 2)))}\\
      \nxt{\bda\\
      \tt{(sqr (* 2 (sqr (qpow 2 1))))}\\
      \nxt{\bda\\
      \tt{(sqr (* 2 (sqr (* 2 (qpow 2 0)))))}\\
      \nxt{\bda\\
      \tt{(sqr (* 2 (sqr (* 2 1))))}\\
      \nxt{\bda\\
      \tt{(sqr (* 2 (sqr 2)))}\\
      \nxt{\bda\\
      \tt{(sqr (* 2 4))}\\
      \nxt{\bda\\
      \tt{(sqr 8)}\\
      \nxt{\bda\\
      \tt{64}}}}}}}}}}
    \end{tabular}
  \end{center}\ \\
  \nxt{Логаритмичен рекурсивен процес}
\end{frame}

\subsection{Дървовидни рекурсивни процеси}

\begin{frame}[fragile]
  \frametitle{Числа на Фибоначи}

  $0, 1, 1, 2, 3, 5, 8, 13, 21, 34, 55, 89, 144, 233, 377, \ldots$\\
  \pause
  \begin{equation*}
    f_n =
    \begin{cases}
      0, &\text{ за }n = 0,\\
      1, &\text{ за }n = 1,\\
      f_{n-1} + f_{n-2}, &\text{ за }n \geq 2.
    \end{cases}
  \end{equation*}
  \pause
\begin{lstlisting}
(define (fib n)
  (cond ((= n 0) 0)
        ((= n 1) 1)
        (else (+ (fib (- n 1)) (fib (- n 2))))))
\end{lstlisting}
  \pause
  $f_{40} = ?$
\end{frame}

\begin{frame}<handout:5>
  \frametitle{Дървовидна рекурсия}

  \begin{center}
    \only<1 |handout:0>{
      \begin{forest} for tree={edge=->}
        [\tt{(fib 2)}
        [\tt{(fib 1)} [\tt 1]] [\tt{(fib 0)}
        [\tt 0]]]
      \end{forest}
    }
    \only<2 |handout:0>{
      \begin{forest} for tree={edge=->}
      [\tt{(fib 3)}
      [\tt{(fib 2)}
      [\tt{(fib 1)} [\tt 1]]
      [\tt{(fib 0)} [\tt 0]]]
      [\tt{(fib 1)} [\tt 1]]]
      \end{forest}
    }
    \only<3 |handout:0>{
      \begin{forest} for tree={edge=->}
      [\tt{(fib 4)}
      [\tt{(fib 3)}
      [\tt{(fib 2)}
      [\tt{(fib 1)} [\tt 1]]
      [\tt{(fib 0)} [\tt 0]]]
      [\tt{(fib 1)} [\tt 1]]]
      [\tt{(fib 2)}
      [\tt{(fib 1)} [\tt 1]]
      [\tt{(fib 0)} [\tt 0]]]]
      \end{forest}
    }
    \only<4 |handout:0>{
      \scriptsize
      \begin{forest} for tree={edge=->}
      [\tt{(fib 5)}
      [\tt{(fib 4)}
      [\tt{(fib 3)}
      [\tt{(fib 2)}
      [\tt{(fib 1)} [\tt 1]]
      [\tt{(fib 0)} [\tt 0]]]
      [\tt{(fib 1)} [\tt 1]]]
      [\tt{(fib 2)}
      [\tt{(fib 1)} [\tt 1]]
      [\tt{(fib 0)} [\tt 0]]]]
      [\tt{(fib 3)}
      [\tt{(fib 2)}
      [\tt{(fib 1)} [\tt 1]]
      [\tt{(fib 0)} [\tt 0]]]
      [\tt{(fib 1)} [\tt 1]]]]
      \end{forest}
    }
    \only<5->{
    \tiny
    \begin{forest} for tree={edge=->,s sep=1pt,inner sep=1pt}
      [\tt{(fib 6)}
      [\tt{(fib 5)}
      [\tt{(fib 4)}
      [\tt{(fib 3)}
      [\tt{(fib 2)}
      [\tt{(fib 1)} [\tt 1]]
      [\tt{(fib 0)} [\tt 0]]]
      [\tt{(fib 1)} [\tt 1]]]
      [\tt{(fib 2)}
      [\tt{(fib 1)} [\tt 1]]
      [\tt{(fib 0)} [\tt 0]]]]
      [\tt{(fib 3)}
      [\tt{(fib 2)}
      [\tt{(fib 1)} [\tt 1]]
      [\tt{(fib 0)} [\tt 0]]]
      [\tt{(fib 1)} [\tt 1]]]]
      [\tt{(fib 4)}
      [\tt{(fib 3)}
      [\tt{(fib 2)}
      [\tt{(fib 1)} [\tt 1]]
      [\tt{(fib 0)} [\tt 0]]]
      [\tt{(fib 1)} [\tt 1]]]
      [\tt{(fib 2)}
      [\tt{(fib 1)} [\tt 1]]
      [\tt{(fib 0)} [\tt 0]]]]]
    \end{forest}
    }
  \end{center}\ \\
  \onslide<6>{Дървовиден рекурсивен процес}
\end{frame}

\begin{frame}[fragile]
  \frametitle{Как да оптимизираме?}

  \textbf{Решение №1: мемоизация}\\
  Да помним вече пресметнатите стойности, вместо да ги смятаме пак.\\
  \pause
  \alert{За ефективна реализация обикновено са нужни странични ефекти.}\\[1em]
  \pause
  \textbf{Решение №2: динамично програмиране}\\
  Строим последователно всички числа на Фибоначи в нарастващ ред.\\
  \pause
  \alert{Нужно е да помним само последните две числа!}\\
  \pause
\begin{lstlisting}
(define (fib n)
  (define (iter i fi fi-1)
    (if (= i n) fi
        (iter (+ i 1) (+ fi fi-1) fi)))
  (if (= n 0) 0
      (iter 1 1 0)))
\end{lstlisting}
\end{frame}

\begin{frame}
  \frametitle{Итеративно генериране на числата на Фибоначи}

  \small
  \begin{center}
    \begin{tabular}{c}
      \nxt{\tt{(fib 7)}\\
      \nxt{\bda\\
      \tt{(iter 1 1 0)}\\
      \nxt{\bda\\
      \tt{(iter 2 1 1)}\\
      \nxt{\bda\\
      \tt{(iter 3 2 1)}\\
      \nxt{\bda\\
      \tt{(iter 4 3 2)}\\
      \nxt{\bda\\
      \tt{(iter 5 5 3)}\\
      \nxt{\bda\\
      \tt{(iter 6 8 5)}\\
      \nxt{\bda\\
      \tt{(iter 7 13 8)}\\
      \nxt{\bda\\
      \tt{13}}}}}}}}}}
    \end{tabular}
  \end{center}
\end{frame}

\end{document}
