\documentclass{beamer}
\usepackage{fprog}

\title{Кортежи и списъци}

\date{6--7 януари 2016 г.}

\begin{document}

\begin{frame}
  \titlepage
\end{frame}

%\includeonlyframes{current}

\section{Кортежи}

\begin{frame}
  \frametitle{Кортежи --- дефиниция}
  
\end{frame}

\begin{frame}
  \frametitle{Кортежи --- примери}
  
\end{frame}

\begin{frame}
  \frametitle{Потребителски типове}
  
\end{frame}

\begin{frame}
  \frametitle{Особености на кортежите}
  
\end{frame}

\begin{frame}
  \frametitle{Кортежи и образци}
  
\end{frame}

\begin{frame}
  \frametitle{Именувани образци}
  
\end{frame}

\section{Списъци}

\subsection{Дефиниция и синтаксис}


\begin{frame}
  \frametitle{Списъци}
  
\end{frame}

\begin{frame}
  \frametitle{Синтаксис за списъци}
  
\end{frame}

\begin{frame}
  \frametitle{Генератори на списъци}
  
\end{frame}

\begin{frame}
  \frametitle{Основни функции за списъци}
  % head, tail, null, length
\end{frame}

\begin{frame}
  \frametitle{Низове}
  
\end{frame}

\subsection{Работа със списъци}

\begin{frame}
  \frametitle{Рекурсия над списъци}
  %++, !!, reverse, elem, notElem
\end{frame}

\begin{frame}
  \frametitle{Образци и списъци}
  
\end{frame}

\begin{frame}
  \frametitle{Полиморфни функции}
\end{frame}

\begin{frame}
  \frametitle{Класове от типове (typeclasses)}
  
\end{frame}

\begin{frame}
  \frametitle{Стандартни класове}
  
\end{frame}

\subsection{Отделяне на списъци}

\begin{frame}
  \frametitle{Отделяне на списъци (list comprehension)}
  
\end{frame}

\begin{frame}
  \frametitle{Отделяне от повече списъци}
  
\end{frame}

\section{Функции над списъци}

\begin{frame}
  \frametitle{Отрязване на списъци}
  % take, drop, splitAt
\end{frame}

\begin{frame}
  \frametitle{Агрегиращи функции}
  % maximum, minimum, product, sum, and, or, concat
\end{frame}

\section{Функции от по-висок ред над списъци}

\begin{frame}
  \frametitle{Трансформация (\tt{map})}
  
\end{frame}

\begin{frame}
  \frametitle{Филтриране (\tt{filter})}
  
\end{frame}

\begin{frame}
  \frametitle{Дясно и ляво свиване (\tt{foldr} и \tt{foldl})}
  
\end{frame}

\begin{frame}
  \frametitle{Свиване на непразни списъци (\tt{foldr1} и \tt{foldl1})}
  
\end{frame}

\begin{frame}
  \frametitle{Сканиране на списъци (\tt{scanl}, \tt{scanr})}
  
\end{frame}


\begin{frame}
  \frametitle{Съшиване на списъци (\tt{zip}, \tt{zipWith})}
  
\end{frame}

\begin{frame}
  \frametitle{Разбиване на списъци (\tt{break}, \tt{span}, \tt{takeWhile}, \tt{dropWhile}) 
  
\end{frame}

\begin{frame}
  \frametitle{Логически квантори (\tt{all}, \tt{any})}
  
\end{frame}

\end{document}

