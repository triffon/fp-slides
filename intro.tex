\documentclass{beamer}
\usepackage[T2A]{fontenc}
\usepackage[utf8]{inputenc}
\usepackage{hyperref}
\usepackage[english,bulgarian]{babel}

\deftranslation[to=bulgarian]{Theorem}{Теорема}

\mode<presentation>
{
%  \usetheme{Hannover}
}

\usepackage{fp}

\title{Какво е функционално програмиране?}

\author{Трифон Трифонов}

\date{\small Функционално програмиране, спец. Информатика, 2015/16 г.}

\subject{LectureSlides}

\begin{document}

\begin{frame}
  \titlepage
\end{frame}

\section*{Императивен и декларативен стил}

\begin{frame}
  \frametitle{Императивен стил}

  Описваме последователно изчислителните стъпки.

  \begin{columns}[t,onlytextwidth]
    \column{0.5\textwidth}

    \textbf{Неструктурирано програмиране}

    \begin{enumerate}
    \item Въведи \tt a, \tt b
    \item Ако \tt{a = b}, към 6.
    \item Ако \tt{a > b}, към 5.
    \item \tt{b $\leftarrow$ b - a}; към 2.
    \item \tt{a $\leftarrow$ b - a}; към 2.
    \item Изведи a
    \item Край
    \end{enumerate}

    \column{0.5\textwidth}

    \textbf{Структурирано програмиране}

    \begin{itemize}
    \item Въведи \tt a, \tt b
    \item Докато \tt{a != b}
      \begin{itemize}
      \item Ако \tt{a > b}
        \begin{itemize}
        \item \tt{a $\leftarrow$ b - a}
        \end{itemize}
      \item В противен случай
        \begin{itemize}
        \item \tt{b $\leftarrow$ b - a}
        \end{itemize}        
      \end{itemize}
    \item Изведи \tt a
    \end{itemize}

  \end{columns}
\end{frame}

\begin{frame}
  \frametitle{Декларативен стил}

  Описваме свойствата на желания резултат.
  \vspace{1em}

  \textbf{Програмиране с ограничения}

  \begin{itemize}
  \item Дадени са \tt a и \tt b.
  \item Търсим \tt d, такова че:
    \begin{itemize}
    \item \tt{1 $\leq$ d $\leq$ a,b}
    \item ``\tt d е делител на \tt a''
    \item ``\tt d е делител на \tt b''
    \item \tt d е възможно най-голямо, където
    \item за дадени \tt x и \tt y казваме, че ``\tt x е делител на \tt y'', ако
    \item намерим такова естествено число \tt k, че
      \begin{itemize}
      \item \tt{1 $\leq$ k $\leq$ y}
      \item \tt k * \tt x = \tt y
      \end{itemize}
    \end{itemize}
  \end{itemize}
\end{frame}

\begin{frame}
  \frametitle{Декларативен стил (2)}

  Описваме свойствата на желания резултат.
  \vspace{1em}

  \textbf{Логическо програмиране}

  \begin{itemize}
  \item Описваме релацията над естествени числа $gcd(a,b,c)$
  \item $\forall a \, gcd(a,a,a)$ (факт)

  \item $\forall a\forall b ( a > b \land \forall c (gcd(a-b,b,c) \rightarrow gcd(a,b,c)))$ (правило)
  \item $\forall a\forall b ( a < b \land \forall c (gcd(a,b-a,c) \rightarrow gcd(a,b,c)))$ (правило)
  \item Дадени са $a, b$
  \item Намери такова $c$, за което $gcd(a,b,c)$
  \end{itemize}

  \pause

  \textbf{Пример:}
  Нека $a = 8, b = 12$. Тогава:

  $gcd(4,4,4) \rightarrow gcd(8,4,4) \rightarrow gcd(8,12,4)$
\end{frame}


\begin{frame}
  \frametitle{Декларативен стил (3)}

  Описваме свойствата на желания резултат.
  \vspace{1em}

  \textbf{Функционално програмиране}

  \begin{itemize}
  \item Функцията над естествени числа $gcd(a,b)$ притежава следните свойства:
  \item $gcd(a,a) = a$
  \item $gcd(a-b,b) = gcd(a,b)$, ако $a > b$
  \item $gcd(a,b-a) = gcd(a,b)$, ако $b > a$
  \item Дадени са $a, b$
  \item Да се пресметне $gcd(a,b)$.
  \end{itemize}

  \pause

  \textbf{Пример:}
  Нека $a = 8, b = 12$.

  $gcd(8,12) = gcd(8,4) = gcd(4,4) = 4$.
\end{frame}

\begin{frame}
  \frametitle{Още един пример}

  Да се намери сумата на квадратите на нечетните числа в списъка \tt l.

  \begin{columns}[t,onlytextwidth]
    \column{0.5\textwidth}
    
    \textbf{Императивен стил}
    \begin{itemize}
    \item Нека \tt{s = 0}.
    \item За \tt i от 1 до \tt{length(l)}:
      \begin{itemize}
      \item Ако \tt{l[i]} е нечетно, то
        \begin{itemize}
        \item \tt{s = s + l[i]$^2$}.
        \end{itemize}
      \end{itemize}
    \item Изведи \tt s.
    \end{itemize}

    \column{0.5\textwidth}

    \textbf{Функционален стил}
    
    \begin{itemize}
    \item От елементите на \tt l:
    \item Избери нечетните
    \item Приложи над резултата функцията $x^2$
    \item Приложи над резултата операцията $+$.
    \end{itemize}
  \end{columns}
\end{frame}

\begin{frame}[fragile]
  \frametitle{Още един пример (2)}
  \textbf{C++:}
\begin{verbatim}
int s = 0;
for(int i = 0; i < sizeof(l); i++)
  if (l[i] % 2 != 0)
    s += l[i] * l[i];
cout << s;
\end{verbatim}
  
  \pause
  \textbf{Scheme:} \tt{(apply + (map square (filter odd? l)))}
  \vspace{1em}

  \pause
  \textbf{Haskell:} \tt{foldr1 (+) [x * x | x <- l, odd x]}
\end{frame}

\section*{Изчислителни модели}

\begin{frame}
  \frametitle{Какво може да се сметне с компютър?}
  
  Нека $f:\mathbb N\to\mathbb N$ е функция над естествени числа.

  Примери: $f(x) = x^2$, $f(x) = x$-тото число на Фибоначи.
  \vspace{2em}
  \pause

  \textbf{Въпрос 1:} Какво означава да изчислим $f$ с компютър?
  \vspace{2em}
  \pause

  \textbf{Въпрос 2:} Какво означава ``алгоритъм'' или ``програма''?
  \vspace{2em}

  \pause

  \textbf{Върпос 3:} Има ли функции, които не могат да бъдат изчислени с компютър?
\end{frame}

\begin{frame}
  \frametitle{Машина на Тюринг}

  \includegraphics[width=\textwidth]{turing.pdf}

  Казваме, че машината \textbf{$M$ изчислява функцията $f_M$}, ако за лента, в която е записано (в двоичен вид) числото $n$, $M$ завършва и записва върху лентата числото $f_M(n)$.

  Ако $M$ не завърши, казваме, че \textbf{$f_M(n)$ не е дефинирана}. 
\end{frame}

\begin{frame}
  \frametitle{Има неизчислими функции!}

  \begin{itemize}[<+->]
  \item Всяка машина на Тюринг може да се кодира като дълго естествено число.
  \item Всяка изчислима функция се изчислява от (поне една) машина.
  \item Следователно, изчислимите функции са не повече от естествените числа (изброимо много).
  \item Но функциите от вида $\mathbb N \to \mathbb N$ са колкото редиците от естествени числа
\ldots
  \item \ldots които са неизброимо много! (защо?).
  \item Следователно, има неизброимо много неизчислими функции. \qed
  \item \alert{Но кои са те?}
  \end{itemize}
\end{frame}

\begin{frame}
  \frametitle{Стоп проблем}

  Нека с $\mt n$ означаваме машината на Тюринг с код $n$.

  Разглеждаме функцията:
  \begin{equation*}
    halts(n) = \mt n\text{ завършва над лента с числото }n.
  \end{equation*}

  \pause

  $halts$ не е изчислима!

  \pause

  Да допуснем, че $halts$ се изчислява от машина на Тюринг $H$.

  Дефинираме нова машина на Тюринг $D$:

  \begin{enumerate}
  \item (тук слагаме всички инструкции на $H$)
  \item[$k+1$.] \tt{IFZERO} $k+3$
  \item[$k+2$.] \tt{JUMP} $k+1$
  \item[$k+3$.] \tt{STOP}
  \end{enumerate}
  
  Нека $D = \mt d$. Завършва ли $D$ над $d$? \qed
\end{frame}

\begin{frame}
  \frametitle{Въпроси за изчислимост}

  Според вас изчислими ли са следните функции?
  \begin{itemize}[<+->]
  \item $f_1(n) = n$ е просто число
  \item $f_2(n) = n$-тото поред просто число
  \item $f_3(n) = n$-тата цифра на числото $\pi$
  \item $f_4(n) = $ има $n$ последователни седмици в числото $\pi$
  \item $f_5(n) = n$ е код на множество от матрици 3x3, които могат да се умножат в някакъв ред, така че да се получи O
  \item $f_6(n) = n$ е код на вярна съждителна формула
  \item $f_7(n) = n$ е код на вярна предикатна формула
  \item $f_8(n) = m$, където $\mt m$ пресмята $f_8$ 
  \item $f_9(n) = $ машините $\mt n$ и $\mt{2n}$ изчисляват еднакви функции
  \end{itemize}
\end{frame}

\begin{frame}
  \frametitle{$\lambda$-смятане}

  Нека разполагаме с изброимо много променливи $x,y,z,\ldots$
  \vspace{1em}

  Три вида $\lambda$-изрази ($E$)
  \begin{itemize}
  \item $x$ (променлива)
  \item $E_1(E_2)$ (апликация, прилагане на функция)
  \item $\lambda x \, E$ (абстракция, конструиране на функция)
  \end{itemize}
  \vspace{1em}

  \pause

  Примери: $\lambda x\, x, \quad (\lambda x\, x)(z), \quad \lambda f\lambda x\, f(f(f(x)))$
  \vspace{1em}

  \pause

  Едно изчислително правило:
  \begin{equation*}
    (\lambda x\,E_1)(E_2) \mapsto E_1[x := E_2].
  \end{equation*}
\end{frame}

\begin{frame}
  \frametitle{Машини на Тюринг = $\lambda$-смятане}

  \begin{theorem}[Alan Turing, 1937]
  Функциите, които могат да се изчислят с машина на Тюринг са точно тези, които могат да се напишат с $\lambda$-израз.
  \end{theorem}

  \pause

  \begin{center}
    \begin{tabular}{|rcl|}
      \hline
      Машини на Тюринг & = & императивен стил за програмиране\\
      $\lambda$-смятане & = & функционален стил за програмиране\\
      \hline
    \end{tabular}
  \end{center}

  \pause

  \textbf{Факт: }Почти всички съвременни езици за програмиране са със същата изчислителна сила като на машините на Тюринг.
  \vspace{1em}

  \pause

  \textbf{Тезис на Church-Turing:} Всяка функция, чието изчисление може да се автоматизира, може да бъде пресметната с машина на Тюринг.
  
\end{frame}

\section*{Особености на функционалното програмиране}

\begin{frame}
  \frametitle{Във функционалното програмиране...}

  ... има:
  \begin{itemize}[<+->]
  \item функции с параметри, (абстракция)
  \item които могат да се прилагат над аргументи, (апликация)
  \item които могат да са други функции (функции от висок ред)
  \item и могат да се дефинират чрез себе си, (рекурсия)
  \end{itemize}
  \vspace{1 em}
  \pause
  ... но няма:
  \begin{itemize}[<+->]
  \item памет
  \item присвояване
  \item цикли
  \item прескачане (goto, break, return)
  \end{itemize}
\end{frame}

\begin{frame}
  \frametitle{Защо функционално програмиране?}

  \begin{itemize}[<+->]
  \item Кратки и ясни програми (изразителност)
  \item Лесна проверка за коректност
  \item При еднакви входни данни връщат един и същ резултат (референциална прозрачност), \pause което позволява...
  \item Избягване на повторно пресмятане на резултати чрез запомняне (мемоизация)
  \item Премахване на части от програмата, които не участват в крайния резултат (мъртъв код)
  \item Пренареждане на програмата за по-ефективно изпълнение (стратегия за оценяване)
  \item Паралелно изпълнение на независими части от програмата (паралелизация)
  \end{itemize}
\end{frame}

\begin{frame}[<1-2>]
  \frametitle{Видове функционални езици}

  \begin{itemize}
  \item според типовата система
    \begin{itemize}
    \item динамично типизирани (стойностите имат тип) \onslide<2->{[\textbf{Scheme}]}
    \item статично типизирани (променливите имат тип) \onslide<2->{[\textbf{Haskell}]}
    \end{itemize}
  \item според страничните ефекти
    \begin{itemize}
    \item нечисти (със странични ефекти) \onslide<2->{[\textbf{Scheme}]}
    \item чисти (без странични ефекти) \onslide<2->{[\textbf{Haskell}]}
    \end{itemize}
  \item според стратегията за оценяване
    \begin{itemize}
    \item стриктно (първо сметни, после предай)  \onslide<2->{[\textbf{Scheme}]}
    \item мързеливо (първо предай, после смятай) \onslide<2->{[\textbf{Haskell}]}
    \end{itemize}
  \end{itemize}

  \pause
\end{frame}

\begin{frame}
  \frametitle{История на функционалното програмиране}

  \begin{itemize}[<+->]
  \item[(1936)] Church и Rosser дефинират $\lambda$-смятането
  \item[(1960)] McCarthy създава първият функционален език LISP
  \item[(1975)] Steele и Sussman създават \textbf{Scheme}, диалект на LISP
  \item[(1977)] Backus (авторът на FORTRAN) популяризира функционалния стил
  \item[(1985)] Turner създава Miranda, първият комерсиален чист функционален език
  \item[(1990)] Публикувана е първата версия на \textbf{Haskell}
  \item[(1990-те)] Функционални елементи започват да се появяват в императивни езици: Python (1991), JavaScript (1995), Ruby (1995), ActionScript (1998)
  \item[(2000-те)] Функционалният стил на програмиране превзема света: Scala (2003), F\# (2005), C\# (2007), Clojure (2007), C++11 (2011), Java 8 (2014)
  \end{itemize}
\end{frame}

\end{document}
