\documentclass{beamer}
\usepackage{fp}

\title{Основни понятия в Scheme}

\date{14 октомври 2015 г.}

\begin{document}

\begin{frame}
  \titlepage
\end{frame}

\section{Въведение в Scheme}

\begin{frame}
  \frametitle{Що за език е Scheme?}

  \begin{itemize}
  \item Създаден през 1975 г. от Guy L.~Steele и  Gerald Jay Sussman
  \item Диалект на LISP, създаден с учебна цел
  \item ``Structure and Interpretation of Computer Programs'', Abelson \& Sussman, MIT Press, 1985.
  \item Минималистичен синтаксис
  \item Най-разпространен стандарт: R$^5$RS
  \end{itemize}
\end{frame}

\begin{frame}
  \frametitle{Програмиране на Scheme}

  \begin{itemize}
  \item Среда за програмиране: DrRacket
  \item Има компилатори и интерпретатори
    \begin{itemize}
    \item Ние ще ползваме интерпретатор
    \end{itemize}
  \item REPL = Read-Eval-Print-Loop
  \item Програма = списък от дефиниции
  \item Изпълнение = оценка на израз
  \end{itemize}
\end{frame}

\begin{frame}
  \frametitle{Синтаксис в Scheme}

  \begin{itemize}
  \item Атоми
    \begin{itemize}
    \item Булеви константи (\tt{\#f}, \tt{\#t})
    \item Числа (\tt{15}, \tt{2/3}, \tt{-1.532})
    \item Низове (\tt{"Scheme"}, \tt{"hi "})
    \item Символи (\tt f, \tt{square}, \tt +, \tt{find-min})
    \end{itemize}
  \item Комбинации\\
    \vspace{1em}
    \tt{\alert({}<израз$_1$> <израз$_2$> \ldots <израз$_n$>\alert)}
  \end{itemize}
\end{frame}

\begin{frame}[fragile]
  \frametitle{Оценки на атоми}

  На всеки израз се дава оценка.
  \begin{itemize}[<+->]
  \item Оценката на булевите константи, числата и низовете са самите те
    \begin{itemize}
    \item \evalsto 5 5
    \item \evalsto{\#t}{\#t}
    \item \evalsto{"scheme"}{"scheme"}
    \end{itemize}
  \item Оценката на символ е стойността, свързана с него
    \begin{itemize}
    \item \evalsto +{\#<procedure:+>}
    \item \evalstoerr a
    \item \tt{(define a 5)}
    \item \evalsto a5
    \end{itemize}
  \end{itemize}
\end{frame}

\begin{frame}
  \frametitle{Основно правило за оценяване}

  Оценка на комбинация \textbf{(основно правило за оценяване)}

  \vspace{1em}
  \begin{equation*}
    \newcommand{\evals}{\text{\small се оценява до}}
    \underbrace{%
      \begin{array}{cccccc}
        (&\text{<израз}_0\text{>}&\text{<израз}_1\text{>}&\ldots&\text{<израз}_n\text{>}&)\\
         &\vert&\vert&&\vert&\\
         &\evals&\evals&&\evals&\\
         &\bda&\bda&&\bda&\\
        (&f&v_1&\ldots&v_n&)
      \end{array}
    }_{%
      \begin{array}{c}
        \vert\\
        \evals\\
        \bda\\
        f(v_1,\ldots,v_n)
      \end{array}}
  \end{equation*}

  \pause

  Ако $f$ не е функция --- \alert{грешка!}
\end{frame}

\begin{frame}[fragile]
  \frametitle{Пример за оценяване на комбинации}
  \begin{equation*}
    \underbrace{%
      \begin{tabular}{cccc}
        \tt{(+}&$\underbrace{\tt{(* 2 3)}}_{}$&$\underbrace{\tt{(- 14 6 5)}}_{}$&\tt)\\
               &\bda&\bda&\\
        \tt{(+}&\tt{6}&\tt{3}&\tt)
      \end{tabular}
    }_{%
      \begin{tabular}{c}
        \bda\\
        \tt 9
      \end{tabular}}
  \end{equation*}

  \vspace{2em}
  \begin{center}
    \evalstoerr {(1 2 3)}
  \end{center}
\end{frame}

\section{Дефиниции в Scheme}

\begin{frame}
  \frametitle{Дефиниция на символи}

  \begin{itemize}[<+->]
  \item \tt{\alert{(define} <символ> <израз>\alert)}
  \item Оценява \tt{<израз>} и свързва \tt{<символ>} с оценката му.
  \item Примери:
    \begin{itemize}
    \item \tt{(define s "Scheme is cool")}
    \item \evalsto s{"Scheme is cool"}
    \item \tt{(define x 2.5)}
    \item \evalsto x{2.5}
    \item \evalsto{(+ x 3.2)}{5.7}
    \item \tt{(define y (+ x 3.2))}
    \item \evalsto{(> y 3)}{\#t}
    \item \evalstoerr{(define z (+ y z))}
    \end{itemize}
  \end{itemize}
\end{frame}

\begin{frame}
  \frametitle{Специални форми}

  \begin{itemize}[<+->]
  \item По основното правило ли се оценява \tt{(define x 2.5)}?
  \item \alert{Не!}
  \item В синтаксиса на Scheme има конструкции, които са изключение от стандартното правило
  \item Такива конструкции се наричат \textbf{специални форми}
  \item \tt{define} е пример за специална форма
  \end{itemize}
\end{frame}

\begin{frame}
  \frametitle{Цитиране}

  \begin{itemize}[<+->]
  \item \tt{\alert{(quote} <израз>\alert)}
  \item Алтернативен запис: \tt{\alert' <израз>}
  \item Оценката на \tt{(' <израз>)} е самият \tt{<израз>}
  \item \textbf{Примери:}
    \begin{itemize}
    \item \evalsto{'2}2
    \item \evalsto{'+}+
    \item \evalsto{'(+ 2 3)}{(+ 2 3)}
    \item \evalsto{(quote quote)}{quote}
    \item \evalstoerr{('+ 2 3)}
    \item \evalstoerr{(/ 2 0)}
    \item \evalsto{'(/ 2 0)}{(/ 2 0)}
    \item \evalsto{'(+ 1 '(* 3 4))}{(+ 1 (quote (* 3 4)))}
    \end{itemize}
  \end{itemize}
\end{frame}

\begin{frame}
  \frametitle{Дефиниция на функции}

  \begin{itemize}[<+->]
  \item \tt{\alert{(define (}{}<функция> \{<параметър>\}\alert{)} <тяло>\alert{)}}
  \item \tt{<функция>} и \tt{<параметър>} са символи
  \item \tt{<тяло>} е \tt{<израз>}
  \item Символът \tt{<функция>} се свързва с поредица от инструкции, които пресмятат \tt{<тяло>} при подадени стойности на \tt{<параметър>}
  \item \textbf{Примери:}
    \begin{itemize}
    \item \tt{(define (square x) (* x x))}
    \item \evalsto{(square 5)}{25}
    \item \tt{(define (1+ k) (+ k 1))}
    \item \evalsto{(square (1+ (square 3)))}{100}
    \item \tt{(define (f x y) (+ (square (1+ x)) (square y) 5)) ;} $(x+1)^2 + y^2 + 5$
    \item \evalsto{(f 2 4)}{30}
    \end{itemize}
  \end{itemize}
\end{frame}

\begin{frame}
  \frametitle{Оценяване на комбинации с дефинирани функции}
  \begin{center}
    \begin{tabular}{c}
      \tt{(f 2 4)}\\
      \pause\nxt{\bda\\
      \tt{(+ (square (1+ 2)) (square 4) 5)}\\
      \nxt{\bda\\
      \tt{(+ (square (+ 2 1)) (square 4) 5)}\\
      \nxt{\bda\\
      \tt{(+ (square 3) (square 4) 5)}\\
      \nxt{\bda\\
      \tt{(+ (* 3 3) (* 4 4) 5)}\\
      \nxt{\bda\\
      \tt{(+ 9 16 5)}\\
      \nxt{\bda\\
      \tt{30}}}}}}}
    \end{tabular}
  \end{center}
\end{frame}

\begin{frame}
  \frametitle{Среди}

  \begin{itemize}
  \item Връзката между символите и техните оценки се записват в речник, който се нарича \textbf{среда}.
  \item Всеки символ има най-много една оценка в дадена среда.
  \item В даден момент могат да съществуват много среди, но само една от която е активна.
  \item \alert{Един символ може да има различни оценки в различни среди.}
  \item При стартиране Scheme по подразбиране работи в \textbf{глобалната среда}.
  \item В глобалната среда са дефинирани символи за стандартни операции и функции.
  \end{itemize}
\end{frame}


\end{document}
