\documentclass{beamer}
\usepackage{fp}

\title[Среди и процеси]{Модел на средите и изчислителни процеси}

\date{21 октомври 2015 г.}

\begin{document}

\begin{frame}
  \titlepage
\end{frame}

\section{Модел на средите}

\begin{frame}
  \frametitle{Среди в Scheme}

  \begin{itemize}[<+->]
  \item Връзката между символите и техните оценки се записват в речник, който се нарича \textbf{среда}.
  \item Всеки символ има най-много една оценка в дадена среда.
  \item В даден момент могат да съществуват много среди.
  \item Символите винаги се оценяват в една конкретна среда.
  \item \alert{Символите могат да има различни оценки в различни среди.}
  \item При стартиране Scheme по подразбиране работи в \textbf{глобалната среда}.
  \item В глобалната среда са дефинирани символи за стандартни операции и функции.
  \end{itemize}
\end{frame}

\begin{frame}
  \frametitle{Пример за среда}

  \begin{columns}[t,onlytextwidth]
    \column{0.5\textwidth}
    {}

    \begin{itemize}[<+->]
    \item \evalstoerr r
    \item \tt{(define r 5)}
    \item \evalsto{(+ r 3)}8
    \item \tt{(define (f x) (* x r))}
    \item \evalsto{(f 3)}{15}
    \item \evalsto{(f r)}{25}
    \end{itemize}

    \column{0.5\textwidth}
    {}

    \begin{envir}{E}
      \\\firstinenv \tt x&:&8
      \only<2->{
        \\\tt r&:&\tt 5
      }
      \only<4->{
        \\\tt f&:&\funcenv x{(* x r)}E
      }
    \end{envir}
  \end{columns}
\end{frame}

\begin{frame}
  \frametitle{Функции и среди}

  \begin{itemize}[<+->]
  \item Всяка функция \tt f пази указател към средата \env E, в която е дефинирана.
  \item При извикване на \tt f:
    \begin{itemize}
    \item създава се нова среда \env{E_1}, която разширява \env E
    \item в \env{E_1} всеки символ означаващ формален параметър се свързва с оценката на фактическия параметър
    \item тялото на $f$ се изпълнява в \env{E_1}
    \end{itemize}
  \end{itemize}
\end{frame}

\begin{frame}
  \frametitle{Дърво от среди}
  \begin{itemize}[<+->]
  \item Всяка среда пази указател към своя ``родителска среда'', която разширява
  \item така се получава дърво от среди
  \item при оценка на символ в дадена среда \env E
    \begin{itemize}
    \item първо се търси оценката му в \env E
    \item ако символът не е дефиниран в \env E, се преминава към родителската среда
    \item при достигане на най-горната среда, ако символът не е дефиниран и в нея се извежда съобщение за грешка
    \end{itemize}
  \end{itemize}
\end{frame}

\begin{frame}
  \frametitle{Извикване на дефинирана функция}

  \begin{columns}[t,onlytextwidth]
    \column{0.5\textwidth}
    {}

    \begin{itemize}[<+->]
    \item \tt{(define r 5)}
    \item \tt{(define a 3)}
    \item \tt{(define (f x) (* x r))}
    \item \begin{tabular}[t]{lc}
            \inenv E&\tt{(f a)}\\
            \nxt{&\bda\\
            \inenv E &\tt{(f 3)}\\
            \nxt{&\bda\\
            \inenv {E_1} &\tt{(* x r)}\\
            \nxt{&\bda\\
            &\tt{15}}}}
          \end{tabular}
        \end{itemize}

    \column{0.5\textwidth}
    {}

    \begin{tabular}{c}
      \begin{envir}{E}
        \\\firstinenv\tt r&:&\tt 5
        \only<2->{\\\tt a&:&\tt 3}
        \only<3->{\\\tt f&:&\funcenv x{(* x r)}E}
      \end{envir}
      \\
      \only<6->{
      \Bigg\uparrow\\
      \begin{envir}{E_1}
        \\\firstinenv\tt x&:&3
      \end{envir}}
    \end{tabular}
  \end{columns}
\end{frame}

\section{Рекурсия}

\subsection{Рекурсивни функции}

\begin{frame}
  \frametitle{Какво е рекурсия?}

  \pause
  Повторение чрез позоваване на себе си
  \vspace{1em}

  \pause
  Рекурсивна функция: дефинира се чрез себе си

  \begin{equation*}
    n! =
    \begin{cases}
      1,&\text{ при }n = 0,\\
      n\cdot (n-1)!,&\text{ при }n > 0.
    \end{cases}
  \end{equation*}
  \vspace{1em}

  \pause
  По-точно: решение на рекурсивно уравнение
  \begin{equation*}
    \onslide<5->{\Gamma(f)(n) :=} f(n) =
    \begin{cases}
      1,&\text{ при }n = 0,\\
      n \cdot f(n-1),&\text{ при }n > 0.
    \end{cases}.
  \end{equation*}

  \pause
  \alert{$n!$ е ``най-малкото'' решение на горното уравнение $\Gamma(f) = f$.}
\end{frame}


\begin{frame}[fragile]
  \frametitle{Най-малка неподвижна точка}

  \begin{theorem}[Knaster-Tarski]
    Ако $\Gamma$ е изчислим оператор, то уравнението $f = \Gamma(f)$ има единствено най-малко решение \pause (\textbf{най-малка неподвижна точка на $\Gamma$}). \pause Нещо повече, решението точно съответства на рекурсивна програма пресмятаща $f$ чрез $\Gamma$.
  \end{theorem}

  \pause

\begin{verbatim}
(define (fact n)
  (if (= n 0) 1
      (* n (fact (- n 1)))))
\end{verbatim}

  \pause

  Кое е \textbf{най-малкото решение} на уравнението $f(x) = 1 + f(x-1)$?

  \pause

\verb#(define (f x) (+ 1 (f (- x 1)))#\\
\evalsto{(f 0)}?

  \pause

  $f$ е ``празната функция'', т.е. $\mathrm{dom}(f) = \emptyset$.

\end{frame}

\subsection{Рекурсивни процеси}

\begin{frame}
  \frametitle{Оценка на рекурсивна функция}

  \begin{center}
    \footnotesize
    \begin{tabular}{c}
      \tt{(fact 4)}\\
      \pause
      \nxt{\bda\\
      \alt<+->{\tt{(* 4 (fact 3))}}{\tt{(if (= 4 0) 1 (* 4 (fact (- 4 1))))}}\\
      \nxt{\bda\\
      \alt<+->{\tt{(* 4 (* 3 (fact 2)))}}{\tt{(* 4 (if (= 3 0) 1 (* 3 (fact (- 3 1)))))}}\\
      \nxt{\bda\\
      \alt<+->{\tt{(* 4 (* 3 (* 2 (fact 1))))}}{\tt{(* 4 (* 3 (if (= 2 0) 1 (* 2 (fact (- 2 1))))))}}\\
      \nxt{\bda\\
      \alt<+->{\tt{(* 4 (* 3 (* 2 (* 1 (fact 0)))))}}{\tt{(* 4 (* 3 (* 2 (if (= 1 0) 1 (* 1 (fact (- 1 1)))))))}}\\
      \nxt{\bda\\
      \tt{(* 4 (* 3 (* 2 (* 1 1))))}\\
      \nxt{\bda\\
      \tt{(* 4 (* 3 (* 2 1)))}\\
      \nxt{\bda\\
      \tt{(* 4 (* 3 2))}\\
      \nxt{\bda\\
      \tt{(* 4 6)}\\
      \nxt{\bda\\
      24}}}}}}}}}
    \end{tabular}
  \end{center}
\end{frame}

\begin{frame}
  \frametitle{Оценка на рекурсивна функция в среда}

  {
  \tiny
  \begin{columns}[t,onlytextwidth]
    \column{0.63\textwidth}
    {}

    \begin{tabular}{lc}
      \nxt{
      \inenv E&\tt{(fact 4)}\\
      \nxt{&\bda\\
      \inenv{E_1}&\alt<+->{\tt{(* 4 (fact 3))}}{\tt{(if (= n 0) 1 (* n (fact (- n 1))))}}\\
      \nxt{&\bda\\
      \inenv{E_2}&\alt<+->{\tt{(* 4 (* 3 (fact 2)))}}{\tt{(* 4 (if (= n 0) 1 (* n (fact (- n 1)))))}}\\
      \nxt{&\bda\\
      \inenv{E_3}&\alt<+->{\tt{(* 4 (* 3 (* 2 (fact 1))))}}{\tt{(* 4 (* 3 (if (= n 0) 1 (* n (fact (- n 1))))))}}\\
      \nxt{&\bda\\
      \inenv{E_4}&\alt<+->{\tt{(* 4 (* 3 (* 2 (* 1 (fact 0)))))}}{\tt{(* 4 (* 3 (* 2 (if (= n 0) 1 (* n (fact (- n 1)))))))}}\\
      \nxt{&\bda\\
      \alt<+->{\inenv{E_4}&\tt{(* 4 (* 3 (* 2 (* 1 1))))}}{\inenv{E_5}&\tt{(* 4 (* 3 (* 2 (* 1 (if (= n 0) 1 (* n (fact (- n 1))))))))}}\\
      \nxt{&\bda\\
      \inenv{E_3}&\tt{(* 4 (* 3 (* 2 1)))}\\
      \nxt{&\bda\\
      \inenv{E_2}&\tt{(* 4 (* 3 2))}\\
      \nxt{&\bda\\
      \inenv{E_1}&\tt{(* 4 6)}\\
      \nxt{&\bda\\
      \inenv E&24}}}}}}}}}}
    \end{tabular}

    \column{0.37\textwidth}
    {}

    \begin{tabular}{*{8}{c@{}}c}
      \multicolumn{9}c{
      \begin{envir}{E}
        \\\firstinenv &&\\[1pt]\hspace{6ex}\tt{fact}&:&\funcenv n\ldots E
      \end{envir}}
      \\
      \multicolumn 2c{\visible<2->{\Bigg\uparrow}}&
      \visible<8->{\Bigg\uparrow}&
      \multicolumn 3c{\visible<4->{\Bigg\uparrow}}&
      \visible<10->{\Bigg\uparrow}&
      \multicolumn 2c{\visible<6->{\Bigg\uparrow}}\\
      \multicolumn 2c{
      \visible<2->{
      \begin{envir}{E_1}
        \\\firstinenv\tt n&:&4
      \end{envir}}}&
      \visible<8->{\Bigg\vert}&
      \multicolumn 3c{
      \visible<4->{
      \begin{envir}{E_2}
        \\\firstinenv\tt n&:&3
      \end{envir}}}&
      \visible<10->{\Bigg\vert}&
      \multicolumn 2c{
      \visible<6->{
      \begin{envir}{E_3}
        \\\firstinenv\tt n&:&2
      \end{envir}}}\\
      \multicolumn 2c{}&
      \visible<8->{\Bigg\vert}&
      \multicolumn 3c{}&
      \visible<10->{\Bigg\vert}&
      \multicolumn 2c{}\\
      &
      \multicolumn 3c{
      \visible<8->{
      \hspace{5ex}
      \begin{envir}{E_4}
        \\\firstinenv\tt n&:&1
      \end{envir}}}&&
      \multicolumn 3c{
      \visible<10->{
      \hspace{2ex}
      \begin{envir}{E_5}
        \\\firstinenv\tt n&:&0
      \end{envir}}}&
    \end{tabular}
  \end{columns}
  }
  \vspace{1em}
  \pause

  Линеен рекурсивен процес
\end{frame}

\subsection{Итеративни процеси}


\begin{frame}[fragile]
  \frametitle{Факториел с цикъл}

  \begin{columns}[t,onlytextwidth]
    \column{0.5\textwidth}
    Факториел на C++
    \vspace{0.5em}

\only<3>\fbox{\tt{int fact(int n)}}\tt{ \{}\\
\verb#  #\only<4>\fbox{\tt{int r = 1;}}\\
\verb#  #\tt{for(\only<5>\fbox{int i = 1}; \only<6>\fbox{i <= n}; \only<7>\fbox{i++})}\\
\verb#    #\only<8>\fbox{\tt{r *= i;}}\\
\verb#  #\only<9>\fbox{\tt{return r;}}\\
\tt{\}}

  \pause

    \column{0.5\textwidth}
    Превод на Scheme
    \vspace{0.5em}

\tt{(define (for n \only<4>\fbox r \only<5>\fbox i)}\\
\verb#  #\tt{(if \only<6>\fbox{(<= i n)}}\\
\verb#      #\tt{(for n \only<8>\fbox{(* r i)} \only<7>\fbox{(+ i 1)})}\\
\verb#      #\tt{\only<9>\fbox r))}\\
\ \\
\only<3>\fbox{\tt{(define (fact n)}}\\
\verb#  #\tt{(for n \only<4>\fbox 1 \only<5>\fbox 1))}
  \end{columns}
\end{frame}

\begin{frame}
  \frametitle{Оценка на итеративен факториел}

  \begin{center}
    \small
    \begin{tabular}{c}
      \tt{(fact 4)}\\
      \nxt{\bda\\
      \tt{(for 4 1 1)}\\
      \nxt{\bda\\
      \alt<+->{\tt{(for 4 1 2)}}{\tt{(if (<= 1 4) (for 4 (* 1 1) (+ 1 1)) 1)}}\\
      \nxt{\bda\\
      \alt<+->{\tt{(for 4 2 3)}}{\tt{(if (<= 2 4) (for 4 (* 1 2) (+ 2 1)) 2)}}\\
      \nxt{\bda\\
      \alt<+->{\tt{(for 4 6 4)}}{\tt{(if (<= 3 4) (for 4 (* 2 3) (+ 3 1)) 6)}}\\
      \nxt{\bda\\
      \alt<+->{\tt{(for 4 24 5)}}{\tt{(if (<= 4 4) (for 4 (* 6 4) (+ 4 1)) 24)}}\\
      \nxt{\bda\\
      \alt<+->{\tt{24}}{\tt{(if (<= 5 4) (for 4 (* 24 5) (+ 5 1)) 24)}}}}}}}}
    \end{tabular}
  \end{center}
  \pause
  \vspace{1em}

  Линеен итеративен процес
\end{frame}

\begin{frame}
  \frametitle{Оценка на итеративен факториел със среди}

  \begin{columns}[t,onlytextwidth]
    \column{0.6\textwidth}
    {}

    \scriptsize
    \begin{tabular}{lc}
      \nxt{\inenv E&\tt{(fact 4)}\\
      &\nxt{\bda\\
      \inenv{E_0}&\alt<+->{\tt{(for 4 1 1)}}{\tt{(for n 1 1)}}\\
      &\nxt{\bda\\
      \inenv{E_1}&\alt<+->{\tt{(for 4 1 2)}}{\tt{(if (<= i n) (for n (* r i) (+ i 1)) r)}}\\
      &\nxt{\bda\\
      \inenv{E_2}&\alt<+->{\tt{(for 4 2 3)}}{\tt{(if (<= i n) (for n (* r i) (+ i 1)) r)}}\\
      &\nxt{\bda\\
      \inenv{E_3}&\alt<+->{\tt{(for 4 6 4)}}{\tt{(if (<= i n) (for n (* r i) (+ i 1)) r)}}\\
      &\nxt{\bda\\
      \inenv{E_4}&\alt<+->{\tt{(for 4 24 5)}}{\tt{(if (<= i n) (for n (* r i) (+ i 1)) r)}}\\
      &\nxt{\bda\\
      \inenv{E_5}&\alt<+->{\tt{24}}{\tt{(if (<= i n) (for n (* r i) (+ i 1)) r)}}}}}}}}}
    \end{tabular}

    \column{0.4\textwidth}
    {}

    \tiny
    \begin{tabular}{*{8}{c@{}}c}
      \multicolumn 9c{
      \visible<2->{
      \begin{envir}{E_0}
        \\\firstinenv \tt n&:&4
      \end{envir}}}
      \\
      \multicolumn 9c{\visible<2->{\big\downarrow}}\\
      \multicolumn 9c{
      \begin{envir}{E}
        \\\firstinenv &&\\[1pt]\tt{fact}&:&\funcenv n{(for n 1 1)}E\\
        \tt{for}&:&\funcenv{n r i}\ldots E
      \end{envir}}
      \\
      \multicolumn 2c{\visible<4->{\big\uparrow}}&
      \visible<10->{\big\uparrow}&
      \multicolumn 3c{\visible<6->{\big\uparrow}}&
      \visible<12->{\big\uparrow}&
      \multicolumn 2c{\visible<8->{\big\uparrow}}\\
      \multicolumn 2c{
      \visible<4->{
      \begin{envir}{E_1}
        \\\firstinenv\tt n&:&4\\
        \tt r&:&1\\
        \tt i&:&1
      \end{envir}}}&
      \visible<10->{
      \begin{tabular}{@{}c@{}}
      \!\!\Bigg\vert\\[0pt]
      \!\!\bigg\vert
      \end{tabular}}&
      \multicolumn 3c{
      \visible<6->{
      \begin{envir}{E_2}
        \\\firstinenv\tt n&:&4\\
        \tt r&:&1\\
        \tt i&:&2
      \end{envir}}}&
      \visible<12->{
      \begin{tabular}{@{}c@{}}
      \!\!\Bigg\vert\\[0pt]
      \!\!\bigg\vert
      \end{tabular}}&
      \multicolumn 2c{
      \visible<8->{
      \begin{envir}{E_3}
        \\\firstinenv\tt n&:&4\\
        \tt r&:&2\\
        \tt i&:&3
      \end{envir}}}\\
      \multicolumn 2c{}&
      \visible<10->{\big\vert}&
      \multicolumn 3c{}&
      \visible<12->{\big\vert}&
      \multicolumn 2c{}\\
      &
      \multicolumn 3c{
      \visible<10->{
      \hspace{5ex}
      \begin{envir}{E_4}
        \\\firstinenv\tt n&:&4\\
        \tt r&:&6\\
        \tt i&:&4
      \end{envir}}}&&
      \multicolumn 3c{
      \visible<12->{
      \hspace{2ex}
      \begin{envir}{E_5}
        \\\firstinenv\tt n&:&4\\
        \tt r&:&24\\
        \tt i&:&5
      \end{envir}}}&
    \end{tabular}

  \end{columns}
\end{frame}

\end{document}
